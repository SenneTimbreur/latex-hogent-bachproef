%============================================================================== 
% Sjabloon poster bachproef 
%============================================================================== 
% Gebaseerd op document class `a0poster' door Gerlinde Kettl en Matthias Weiser 
% Aangepast voor gebruik aan HOGENT door Jens Buysse en Bert Van Vreckem

\documentclass[a0,portrait]{hogent-poster}

% Info over de opleiding
\course{Bachelorproef}
\studyprogramme{Toegepaste informatica}
\academicyear{2024-2025}
\institution{Hogeschool Gent, Valentin Vaerwyckweg 1, 9000 Gent}

% Info over de bachelorproef
\title{Grenzen van de low-code tool Mendix}
\author{Senne Timbreur}
\email{senne.timbreur@student.hogent.be}
\supervisor{Mevr. F. Spriet}
\cosupervisor{Dhr. B. Claeys (apvine)}

\specialisation{Mobile and Enterprise Development}
\keywords{Mendix, low-code, high-code, beslissingskader}

\begin{document}
    \maketitle
    
    \begin{abstract}
        Deze bachelorproef onderzoekt de grenzen van het low-codeplatform Mendix bij het ontwikkelen van zakelijke applicaties en bepaalt in welke situaties high-codeontwikkeling de voorkeur verdient. Op basis van interviews, projectanalyses en een hands-on vergelijking tussen een applicatie gebouwd in Mendix en in JavaScript/React, werd nagegaan waar Mendix excelleert en waar het tekortschiet. De resultaten tonen aan dat Mendix bijzonder geschikt is voor standaardprocessen, snelle prototyping en projecten met korte doorlooptijden, maar dat het platform beperkt is bij complexe integraties of geavanceerde maatwerkvereisten. De studie benadrukt het belang van een formeel beslissingskader dat zowel technische als organisatorische aspecten integreert, om zo onderbouwde technologiekeuzes te maken en de kwaliteit en efficiëntie van IT-projecten te bevorderen.
    \end{abstract}
    
    \begin{multicols}{2}
        
        \section{Introductie}
        Low-codeplatformen zoals Mendix winnen snel aan populariteit binnen organisaties die hun digitale transformatie willen versnellen. Tijdens mijn stage bij apvine kreeg ik de kans om Mendix hands-on te verkennen. Deze ervaring vormde de aanleiding voor dit onderzoek, dat de sterktes en beperkingen van low-code ontwikkeling binnen een professionele context onderzoekt, met een focus op de afweging tussen low-code en high-code.
        
        \section{Onderzoeksvraag en Doelstelling}
        
        \textbf{Hoofdonderzoeksvraag:} \\
        Wat zijn de beperkingen van de low-code-tool Mendix bij het ontwikkelen van zakelijke applicaties, en welke criteria bepalen wanneer een overstap naar high-code ontwikkeling noodzakelijk wordt?
        
        Het doel is het ontwikkelen van een beslissingskader dat projectmanagers en softwarearchitecten ondersteunt bij het maken van gefundeerde keuzes.
        
        \section{Methodologie}
        Het onderzoek bestaat uit:
        \begin{center}
            \begin{minipage}{1\linewidth}
                \centering
                \includegraphics[height=0.3\textheight, keepaspectratio]{Methodologie.png}
                \captionof{figure}{Methodologie die gehanteerd werd voor het onderzoek}
            \end{minipage}
        \end{center}

        
        \section{Experimenten}
        Praktisch onderzoek werd gedaan door Mendix en JavaScript/React applicaties te vergelijken. Hieronder enkele voorbeelden van de praktische testen:
        
        \begin{center}
            \begin{minipage}{0.48\linewidth}
                \centering
                \scriptsize
                \renewcommand{\arraystretch}{1.2}
                \begin{tabular}{|p{0.18\linewidth}|p{0.15\linewidth}|p{0.15\linewidth}|p{0.15\linewidth}|p{0.2\linewidth}|}
                    \hline
                    \textbf{Metriek} & \textbf{React} & \textbf{Mendix} & \textbf{Verschil} & \textbf{Beste}\\
                    \hline
                    \textbf{FCP} & 11.8s & 13.8s & -2.0s & React \\
                    \hline
                    \textbf{LCP} & 22.8s & 17.8s & +4.9s & Mendix \\
                    \hline
                    \textbf{TBT} & 497ms & 363ms & +134ms & Mendix \\
                    \hline
                    \textbf{Speed Index} & 11.8s & 13.8s & -2.0s & React \\
                    \hline
                    \textbf{CLS} & 0 & 0 & Gelijk & Gelijk \\
                    \hline
                    \textbf{Score} & 43 & 47 & -4 & Mendix \\
                    \hline
                \end{tabular}
                \captionof{table}{Vergelijking van laadtijd}
            \end{minipage}
            \hspace{0.02\linewidth}
            \begin{minipage}{0.48\linewidth}
                \centering
                \scriptsize
                \renewcommand{\arraystretch}{1.2}
                \begin{tabular}{|p{0.22\linewidth}|p{0.15\linewidth}|p{0.22\linewidth}|p{0.15\linewidth}|p{0.22\linewidth}|}
                    \hline
                    \textbf{Fase} & \textbf{Duur (Mendix)} & \textbf{Fase} & \textbf{Duur \newline (React)} & \textbf{Activiteit} \\
                    \hline
                    Initial \newline Animation & $\sim$20 ms & Eventverwerk- ing & $\sim$4 ms & UI-reactie / klikverwerk- ing \\
                    \hline
                    Task \newline Processing & $\sim$20 ms & XHR-verzoek & $\sim$30 ms & Servertijd en netwerkverkeer \\
                    \hline
                    Function \newline Execution & $\sim$40 ms & Responsver- werking & $\sim$20 ms & Verwerking van antwoorddata \\
                    \hline
                    Rendering + GC & $\sim$90 ms & Visuele \newline rendering & $\sim$46 ms & UI-update, schermrendering \\
                    \hline
                    \textbf{Totale \newline Reactietijd} & \textbf{$\sim$170 ms} & \textbf{Totale \newline Reactietijd} & \textbf{$\sim$100 ms} &  \\
                    \hline
                \end{tabular}
                \captionof{table}{Reactietijd per fase: Mendix vs. React}
            \end{minipage}
        \end{center}
        
       \section{Resultaten}
       De experimenten toonden aan dat beide technologieën sterke en zwakke punten hebben, afhankelijk van de context. Deze bevindingen worden hieronder samengevat.
        \begin{itemize}
            \item Mendix is ideaal voor standaard bedrijfsprocessen, MVP's en projecten met korte deadlines.
            \item Beperkingen ontstaan bij geavanceerde integraties, complexe businesslogica of de nood aan technische controle.
            \item Gebrek aan een gestructureerd beslissingsproces leidt tot vertragingen, frictie in teams en verhoogde kosten.
        \end{itemize}
        Op basis van deze resultaten werd een conclusie gevormd over wanneer Mendix al dan niet geschikt is, en waarom een beslissingskader noodzakelijk blijkt.
        
        \section{Technologiekeuze in de praktijk}
        
        Een beslissingsmatrix werd ontwikkeld om objectief te bepalen of een project beter geschikt is voor low-code (Mendix), high-code, of een hybride oplossing. Elke dimensie wordt gescoord op een schaal van 1 (laag) tot 5 (hoog).
        
        \begin{center}
            \begin{minipage}{0.6\linewidth}
                \centering
                \scriptsize
                \renewcommand{\arraystretch}{1.2}
                \begin{tabular}{|p{3cm}|p{5cm}|p{1.2cm}|p{4.3cm}|}
                    \hline
                    \textbf{Dimensie} & \textbf{Vraag} & \textbf{Score (1–5)} & \textbf{Toelichting} \\
                    \hline
                    \textbf{Tijd \& Scope} & Moet er binnen enkele weken een MVP live zijn? & & 1 = Ja, 5 = Nee \\
                    \cline{2-4}
                    & Is het project iteratief/veranderlijk (agile)? & & 1 = Veranderlijk, 5 = Vast \\
                    \hline
                    \textbf{Technische Vereisten} & Is de businesslogica complex of veranderlijk? & & 1 = Simpel, 5 = Complex \\
                    \cline{2-4}
                    & Zijn real-time functionaliteiten vereist? & & 1 = Nee, 5 = Ja \\
                    \hline
                    \textbf{Teamcapaci- teit} & Is er voldoende high-code capaciteit? & & 1 = Beperkt, 5 = Ruim \\
                    \cline{2-4}
                    & Zijn domeinexperten betrokken bij de ontwikkeling? & & 1 = Niet, 5 = Actief \\
                    \hline
                    \textbf{Integratie- behoefte} & Betreft het standaard systemen (bv. SAP)? & & 1 = Ja, 5 = Nee \\
                    \cline{2-4}
                    & Gaat het om complexe/niet-standaard API's? & & 1 = Nee, 5 = Ja \\
                    \hline
                    \textbf{UX/Design} & Moet de UI sterk afwijken van standaardcomponenten? & & 1 = Nee, 5 = Ja \\
                    \cline{2-4}
                    & Is intensieve samenwerking met UX-designers nodig? & & 1 = Beperkt, 5 = Intensief \\
                    \hline
                    \textbf{Beveiliging \& Compliance} & Zijn er strikte eisen (GDPR, audits)? & & 1 = Ja, 5 = Nee \\
                    \cline{2-4}
                    & Wordt gevoelige of gereguleerde data verwerkt? & & 1 = Ja, 5 = Nee \\
                    \hline
                \end{tabular}
                \captionof{table}{Beslissingsmatrix voor technologiekeuze}
            \end{minipage}
            \hfill
            \begin{minipage}{0.35\linewidth}
                \scriptsize
                \textbf{Interpretatie van de totaalscore:} \\
                \vspace{0.5em}
                De som van alle scores geeft een indicatie van de best passende ontwikkelaanpak. \\
                
                \vspace{0.5em}
                \begin{itemize}
                    \item \textbf{12–27 punten:} Mendix aanbevolen \\
                    → Voor snelle, gestandaardiseerde projecten
                    \item \textbf{28–44 punten:} Hybride oplossing \\
                    → Combineer Mendix en high-code (bijv. voor UI)
                    \item \textbf{45–60 punten:} High-code aanbevolen \\
                    → Bij complexe eisen, maatwerk en diepgaande controle
                \end{itemize}
            \end{minipage}
        \end{center}
        
        \section{Conclusie}
        Dit onderzoek toont aan dat Mendix waardevolle voordelen biedt voor standaardprocessen en snelle prototyping, maar beperkt is bij complexe integraties en geavanceerde maatwerkvereisten. Het ontwikkelde beslissingskader biedt een objectieve methode om technologiekeuzes te onderbouwen en kan organisaties helpen kostbare hertoewijzingen te vermijden. Een formeel beslissingsproces verhoogt niet alleen de projectefficiëntie, maar ook het draagvlak voor technologische keuzes binnen IT-teams.
        
        \section{Toekomstig onderzoek}
        Het beslissingskader kan verder worden uitgebreid met kwantitatieve kostenanalyses en longitudinale studies naar de onderhoudbaarheid van Mendix-applicaties. Daarnaast is onderzoek naar hybride architecturen en sectorspecifieke validatie binnen verschillende industrieën waardevol. De ontwikkeling van een geautomatiseerd assessment-tool zou de praktische toepasbaarheid van het framework aanzienlijk kunnen verhogen.
        
    \end{multicols}
    
\end{document}
