%==============================================================================
% Sjabloon onderzoeksvoorstel bachproef
%==============================================================================
% Gebaseerd op document class `hogent-article'
% zie <https://github.com/HoGentTIN/latex-hogent-article>

% Voor een voorstel in het Engels: voeg de documentclass-optie [english] toe.
% Let op: kan enkel na toestemming van de bachelorproefcoördinator!
\documentclass{hogent-article}


% Invoegen bibliografiebestand
\addbibresource{voorstel.bib}

% Informatie over de opleiding, het vak en soort opdracht
\studyprogramme{Professionele bachelor toegepaste informatica}
\course{Bachelorproef}
\assignmenttype{Onderzoeksvoorstel}
% Voor een voorstel in het Engels, haal de volgende 3 regels uit commentaar
% \studyprogramme{Bachelor of applied information technology}
% \course{Bachelor thesis}
% \assignmenttype{Research proposal}

\academicyear{2024-2025}

\title{De grenzen van de low-code-tool Mendix bij het ontwikkelen van zakelijke applicaties.}

%Wat zijn de grenzen van de low-code-tool Mendix bij het ontwikkelen van zakelijke applicaties, en wanneer is high-code-ontwikkeling meer geschikt?

\author{Senne Timbreur}
\email{senne.timbreur@student.hogent.be}


\supervisor[Co-promotor]{J. Alexander (Apvine, \href{mailto:Jodie.Alexander@apvine.com}{Jodie.Alexander@apvine.com})}

% Binnen welke specialisatierichting uit 3TI situeert dit onderzoek zich?
% Kies uit deze lijst:
%
% - Mobile \& Enterprise development
% - AI \& Data Engineering
% - Functional \& Business Analysis
% - System \& Network Administrator
% - Mainframe Expert
% - Als het onderzoek niet past binnen een van deze domeinen specifieer je deze
%   zelf
%
\specialisation{Mobile \& Enterprise development}
\keywords{Mendix, low-code, high-code}

\begin{document}
    
    \begin{abstract}
        In het digitale landschap van vandaag zoeken bedrijven steeds naar manieren om zakelijke applicaties efficiënter te ontwikkelen. Low-code ontwikkelplatformen, zoals Mendix, zijn de laatste jaren sterk gegroeid als een vervanging voor traditionele high-code ontwikkeling. Deze platformen beloven snellere ontwikkelingstijden en toegankelijkheid door visuele interfaces en herbruikbare componenten. Er is echter nog onduidelijkheid over wat de grenzen van de low-code-tool Mendix zijn bij het ontwikkelen van zakelijke applicaties, en wanneer high-code-ontwikkeling meer geschikt is.
        \\
        Dit onderzoek duikt diep in de technische mogelijkheden en beperkingen van Mendix om deze centrale vraag te beantwoorden. Door de complexiteit van deze vraag te ontrafelen, zal de studie een genuanceerd beeld schetsen van de huidige stand van low-code technologie.
        \\
        Een gemengde onderzoeksmethodologie zal worden gehanteerd, bestaande uit een diepgaande literatuurstudie, praktische experimenten en interviews met ervaren ontwikkelaars. Deze veelzijdige onderzoeksaanpak stelt ons in staat om de technische grenzen van Mendix grondig te analyseren, de prestaties in verschillende scenario's te evalueren en concrete richtlijnen te formuleren voor organisaties die overwegen low-code te implementeren.
        \\
        Het onderzoek zal een gedetailleerde analyse geven van de technische beperkingen van Mendix en use cases identificeren waarin de ontwikkeling van high-code meer geschikt is. Het zal ook een beslissingskader bieden, zodat bedrijven en andere instellingen kunnen kiezen tussen low-code en high-code methoden. Uiteindelijk zal het richtlijnen bevatten voor het optimaal combineren van de twee methoden om de beste resultaten te behalen.
        \\
        Gezien de groeiende adoptie van low-code platforms is er dringend behoefte aan objectief onderzoek naar hun mogelijkheden en beperkingen. Deze bachelorproef helpt om deze behoefte aan kennis te vervullen en biedt handige adviezen voor organisaties die overwegen om low-code of traditionele ontwikkelmethoden toe te passen.
    \end{abstract}
    
    \tableofcontents
    
    % De hoofdtekst van het voorstel zit in een apart bestand, zodat het makkelijk
    % kan opgenomen worden in de bijlagen van de bachelorproef zelf.
    %---------- Inleiding ---------------------------------------------------------

\section{Inleiding}%
\label{sec:inleiding}

Apvine, een toonaangevend IT-consultancybedrijf, richt zich op het creëren van applicaties met low-code platforms, voornamelijk Mendix. Deze aanpak is zeer effectief gebleken voor de meeste van hun projecten en maakt snelle ontwikkeling en implementatie mogelijk. Toch zijn er situaties waarin projecten uitdagingen bieden die de grenzen van low-code platforms opzoeken. Denk hierbij aan ingewikkelde bedrijfslogica, intensieve real-time verwerking of complexe integraties met legacy-systemen, die de mogelijkheden van een puur low-code methode kunnen overstijgen.
\\
\\
In dergelijke situaties kan het noodzakelijk zijn om over te stappen op een hybride of high-code methode. Het is echter niet eenvoudig om te beslissen wanneer deze overgang moet gebeuren. Zonder duidelijke richtlijnen loopt Apvine het risico op vertragingen, hogere uitgaven en ontevreden klanten door late of onverwachte aanpassingen in de ontwikkelingsstrategie.
Om deze valkuilen te vermijden, richt dit onderzoek zich op de centrale vraag:
“Wat zijn de grenzen van de low-code-tool Mendix bij het ontwikkelen van zakelijke applicaties, en wanneer is high-code-ontwikkeling meer geschikt?”
\\
\\
Dit onderzoek heeft als doel een raamwerk voor besluitvorming te ontwikkelen dat projectmanagers en architecten van Apvine ondersteunt bij het beoordelen of ze moeten blijven met low-code of overstappen naar high-code voor een specifiek project. Het raamwerk zal worden gebaseerd op:
\begin{itemize}
    \item Systematische analyse van eerdere projecten om gemeenschappelijke succesfactoren en uitdagingen te identificeren.
    \item Richtlijnen voor het implementeren van hybride methoden om zowel flexibiliteit als complexiteit te combineren.
    \item Een hulpmiddel dat beslissingen ondersteunt tijdens de fases van pre-sales en planning.
\end{itemize}
Om de centrale vraag te beantwoorden, worden de volgende deelvragen onderzocht:

\textbf{Probleemdomein:}
\begin{enumerate}
    \item Wat zijn de gevolgen van het ontbreken van een beslissingskader voor de keuze tussen low-code en high-code bij Apvine?
    \item Welke problemen ontstaan er in projecten door het gebrek aan richtlijnen voor de keuze tussen ontwikkelmethoden?
    \item Wat zijn de huidige criteria die Apvine gebruikt bij het kiezen tussen low-code en high-code ontwikkeling?
    \item Welke knelpunten ervaren projectmanagers en architecten bij het maken van de keuze tussen low-code en high-code?
\end{enumerate}


\textbf{Oplossingsdomein:}
\begin{enumerate}
    \setcounter{enumi}{4}
    \item  Wat zijn de technische beperkingen van Mendix bij het omgaan met complexe bedrijfslogica, intensieve real-time verwerking en integraties met legacy-systemen?
    \item In welke specifieke scenario's binnen Mendix-projecten kan een hybride of high-code oplossing nodig zijn, en wat zijn de triggers voor het maken van deze overstap?
    \item  Hoe kunnen projectomvang, tijdslijnen en klantvereisten de beslissing beïnvloeden om wel of niet met Mendix (low-code) te werken, of over te schakelen naar high-code? 
    \item Welke lessen kunnen worden getrokken uit de ervaring van eerdere projecten bij Apvine waarin Mendix werd ingezet, en hoe kunnen deze inzichten helpen bij het bepalen wanneer een hybride of high-code oplossing nodig is?
\end{enumerate}

Het beslissingskader dat uit dit onderzoek voortkomt, biedt een gestructureerde aanpak om organisaties te helpen bij het maken van een weloverwogen keuze tussen low-code en high-code ontwikkelmethoden. Met dit kader zijn bedrijven zoals Apvine in staat om niet alleen effectiever te plannen, maar ook de risico's van late veranderingen in projecten te verlagen, de ontwikkelingskosten effectiever te beheren en hoogwaardige oplossingen te bieden die voldoen aan de unieke wensen van hun klanten.
\\
\\
\\

%---------- Stand van zaken ---------------------------------------------------

\section{Literatuurstudie}%
\label{sec:literatuurstudie}
\subsection{Low-code en Mendix}
Het landschap van digitale klantervaringen evolueert snel, aangewakkerd door de groeiende verwachtingen van klanten en de behoefte voor bedrijven om steeds consistente en hoogwaardige ervaringen te bieden. Zoals beschreven in de tekst “Deliver Standout Digital Customer Experiences with Low-Code” door het bedrijf \textcite{Mendix}, worden bedrijven geconfronteerd met aanzienlijke uitdagingen om aan deze eisen te voldoen vanwege de complexiteit van traditionele ontwikkelpraktijken en beperkte middelen.
\\
Ongelijksoortige processen en afzonderlijke oplossingen leiden vaak tot onsamenhangende ervaringen die niet aansluiten bij de wensen van de klant \autocite{Mendix}. Ondernemingen moeten worstelen met het beheer van meerdere complexe technologieën, afzonderlijke applicaties en een vertraagde time-to-market. Dit alles hindert hen erin om zich snel aan te passen en de ervaringen te bieden die klanten verwachten .
\\
Low-code ontwikkelplatforms zijn aangetoond als een effectieve oplossing voor deze problemen, waardoor bedrijven de controle kunnen nemen over de klantervaring. Low-code platformen maken zowel technische als niet-technische teams mogelijk om samen te werken aan de snelle ontwikkeling van multi-ervaringstoepassingen door snelle ontwikkelingskansen en een klantgerichte benadering te bieden.
\\
Met name het low-code Mendix-platform is ontworpen om bedrijven te ondersteunen bij digitalisering en om consistente, toekomstbestendige contactpunten voor klanten te creëren voor diverse kanalen en apparaten. Dankzij functies zoals automatisering, AI-ondersteuning en cloud-native schaalbaarheid stelt Mendix bedrijven in staat om snel applicaties te bouwen en te implementeren, terwijl ze tegelijkertijd flexibel en consistent blijven. ~\autocite{Mendix}

\subsection{Low-code use cases}
Low-code ontwikkelplatforms maken het voor organisaties mogelijk om een breed scala aan bedrijfsapplicaties te creëren, zodat ze hun bedrijfsprocessen kunnen faciliteren en de gebruikerservaringen kunnen optimaliseren.
Zo zijn er veelvoorkomende use cases:
\begin{itemize}
    \item Legacy modernisering 
    \\
    Met low-code hebben bedrijven de mogelijkheid om legacy-systemen te combineren met nieuwere technologieën, bestaande kansen te vergroten of verouderde systemen te vervangen om in te spelen op veranderende bedrijfsbehoeften. Banco de Occidente maakte gebruik van een low-code platform om hun verouderde systemen te integreren en hun processen te verbeteren, wat leidde tot een betere ervaring voor zowel klanten als medewerkers \autocite{Bunce2024}.
    \item Portalen
    \\
    Low-code stelt gebruikers in staat om op maat gemaakte, webgebaseerde portals te creëren die hen van belangrijke informatie en acties voorzien, wat de efficiëntie en gebruikerservaring ten goede komt. DHL Group heeft een platform ontwikkeld voor het beheer van leveranciersstamgegevens, gebaseerd op een low-code platform, waarmee het werk wordt geautomatiseerd. \autocite{Bunce2024} .
    \item Mobiele apps
    \\
    Low-code applicaties functioneren op diverse apparaten zonder dat er hercodering vereist is, waardoor gebruikers toegang krijgen tot het apparaat dat ze verkiezen. Super Bock Group maakte gebruik van low-code voor het ontwikkelen van een mobiel goedkeuringsproces voor aankoopaanvragen, wat resulteerde in kortere responstijden \autocite{Bunce2024a}.
    \item Integratie
    \\
    Low-code platforms maken het mogelijk om verschillende systemen en gegevensbronnen te integreren, zodat gebruikers in één interface toegang krijgen tot informatie. Zo maakte bijvoorbeeld Unilever gebruik van een low-code platform om hun \\SAP-organisaties en prijsstrategieën voor \\klanten te integreren, wat resulteerde in een hogere efficiëntie en precisie \autocite{Bunce2024}.
\end{itemize}
Dit zijn slechts een paar voorbeelden van de veelzijdigheid en brede toepassing van low-code ontwikkelplatformen in diverse industrieën en gebruikssituaties. Bij een volledige literatuurstudie kan/zal dit verder uitgewerkt worden.

\subsection{Beperkingen van low-code}
Hoewel low-code ontwikkelplatformen veel voordelen opleveren bij het versnellen van de applicatielevering, hebben ze ook enkele opvallende nadelen waar organisaties rekening mee moeten houden. Het artikel “9 Low Code Limitations in 2024 to Know About” \autocite{Malak2024} beschrijft verschillende belangrijke beperkingen die verband houden met low-code methoden. Deze beperkingen omvatten:

\begin{itemize}
    \item Een tekort aan controle over de automatisch genereerde code kan een obstakel vormen voor ingewikkelde applicaties die codeoptimalisatie vereisen.
    \item Potentiële vendor lock-in door eigen frameworks of programmeertalen maakt het lastig om van platform te veranderen of met andere systemen te integreren.
    \item Er zijn beperkte mogelijkheden voor aanpassing, omdat low-code platforms beter geschikt zijn voor algemene functies dan voor ingewikkelde, specifieke vereisten binnen een bepaalde sector.
    \item Beveiligingsproblemen ontstaan doordat low-code ontwikkeling sommige beveiligingsaspecten kan over het hoofd zien, vooral in streng gereguleerde sectoren.
    \item Complexiteit van integratie komt voort uit het feit dat low-code platforms mogelijk niet perfect kunnen integreren met de huidige infrastructuur van een organisatie en externe diensten.
    \item Beperkingen in de schaalbaarheid, omdat low-code applicaties problemen kunnen ondervinden bij het aanpassen aan de groeiende vraag van gebruikers en het volume aan gegevens.
\end{itemize}
De geschiktheid van low-code voor een specifiek project zal uiteindelijk afhankelijk zijn van elementen zoals de ingewikkeldheid van de applicatie, de technische expertise van de organisatie, de behoefte aan beveiliging en naleving, evenals de langetermijnschaalbaarheidseisen. Door de in dit artikel genoemde beperkingen grondig te beoordelen, kunnen bedrijven beter geïnformeerde keuzes maken over het gebruik van low-code ontwikkeling en wanneer traditionele codering beter aansluit.

\subsection{Low-code versus high-code}
Hoewel ze fundamenteel van elkaar verschillen, kunnen low-code en high-code ontwikkelmethoden elkaar aanvullen om aan diverse projecteisen te voldoen. High-code ontwikkeling levert ongeëvenaarde controle, flexibiliteit en de optie om complexe, aangepaste toepassingen vanaf nul te ontwikkelen, wat het perfect maakt voor grote of ingewikkelde projecten die afhankelijk zijn van geavanceerde functies en integraties. Low-code-platforms vergemakkelijken het ontwikkelproces met visuele hulpmiddelen en kant-en-klare componenten, wat ervoor zorgt dat prototypes en de ontwikkeling van middelmatig complexe toepassingen sneller verlopen.
\\
\\
Door deze methoden in evenwicht te brengen, kunnen bedrijven hun productiviteit maximaliseren, de time-to-market voor bepaalde projecten verkorten en ervaren ontwikkelaars inzetten waar ze het meest nodig zijn, zoals \textcite{Ballejos2024} benadrukt in uitgebreide overzicht van de verschillen en mogelijke wisselwerking tussen deze methodologieën.


% Voor literatuurverwijzingen zijn er twee belangrijke commando's:
% \autocite{KEY} => (Auteur, jaartal) Gebruik dit als de naam van de auteur
%   geen onderdeel is van de zin.
% \textcite{KEY} => Auteur (jaartal)  Gebruik dit als de auteursnaam wel een
%   functie heeft in de zin (bv. ``Uit onderzoek door Doll & Hill (1954) bleek
%   ...'')


%---------- Methodologie ------------------------------------------------------
\section{Methodologie}%
\label{sec:methodologie}
Dit onderzoek is opgedeeld in twee fasen: eerst het in kaart brengen van het probleemdomein (het ontbreken van een beslissingskader), gevolgd door onderzoek naar het oplossingsdomein (de ontwikkeling van het kader).
\subsection{Fase 1: Analyse van het probleemdomein}
Om een grondig inzicht te krijgen in het probleemdomein worden de volgende onderzoeksmethoden gebruikt:
\subsubsection{Analyse van historische projectdata}
Om deelvragen 1 en 2 te beantwoorden ("Wat zijn de gevolgen van het ontbreken van een \\ beslissingskader?" \; \hbox{en} \, "Welke problemen ontstaan er in projecten?"), wordt een gestructureerde analyse uitgevoerd van afgesloten projecten. Deze analyse omvat:
\begin{itemize}
    \item Vergelijking van initiële projectschattingen versus werkelijke uitkomsten om impact op kosten en doorlooptijd te kwantificeren
    \item Analyse van projectdocumentatie om momenten te identificeren waar keuzes tussen low-code en high-code tot problemen leidden
    \item Inventarisatie van situaties waarin late overschakeling naar alternatieve ontwikkelmethoden nodig was
    \item Evaluatie van de financiële en tijdsimpact van deze late aanpassingen
\end{itemize}
\subsubsection{Interviews met experts}
Voor het beantwoorden van deelvragen 3 en 4 ("Wat zijn de huidige criteria?" \; \hbox{en} \, "Welke knelpunten ervaren projectmanagers?") worden diepte-interviews gehouden met:
\begin{itemize}
    \item Projectmanagers over hun huidige besluitvormingsproces
    \item Architecten over hun ervaringen met technologie-keuzes
    \item Pre-sales consultants over hun aanpak bij het inschatten van projectgeschiktheid
    \item Ontwikkelaars over de uitdagingen die zij ervaren bij technologie-keuzes
\end{itemize}
\subsubsection{Documentatieonderzoek}
Om deelvraag 3 verder te onderbouwen wordt bestaande interne documentatie geanalyseerd:
\begin{itemize}
    \item Huidige richtlijnen en procedures voor projectaanpak
    \item Pre-sales documentatie en offertes
    \item Project kick-off documenten
    \item Architectuurbeslissingen en design documents
\end{itemize}
\subsection{Fase 2: Onderzoek naar het oplossingsdomein}
Na het volledig in kaart brengen van het probleem, richt het onderzoek zich op het ontwikkelen van een oplossing middels:
\subsubsection{Literatuuroverzicht}
Een diepgaande analyse van bestaande documentatie over Mendix en andere low-code platforms voor het beantwoorden van deelvraag 5 en 7:
\begin{itemize}
    \item Officiële Mendix productgidsen
    \item Casestudies van derden en rapporten uit de industrie
    \item Online forums en ontwikkelaarsgemeenschappen
    \item Academische publicaties over low-code ontwikkeling
\end{itemize}
\subsubsection{Praktisch onderzoek}
Uitvoering van praktische tests voor deelvraag 6 om de grenzen van het Mendix-platform te onderzoeken:
\begin{itemize}
    \item Schaalbaarheid en prestaties
    \item Integratiemogelijkheden
    \item Aanpasbaarheid en uitbreidbaarheid
    \item Ontwikkelingssnelheid
    \item Beveiliging en compliance
\end{itemize}
\subsubsection{Ontwikkeling beslissingskader}
Voor het beantwoorden van deelvraag 8 ("Welke best practices en richtlijnen voor de keuze tussen Mendix en traditionele ontwikkeling kunnen worden gedestilleerd uit de projectervaringen en literatuur?") wordt op basis van alle verzamelde inzichten een gestructureerd beslissingskader ontwikkeld dat projectmanagers en architecten ondersteunt bij het maken van weloverwogen keuzes over de inzet van Mendix versus traditionele ontwikkelmethoden.



%---------- Verwachte resultaten ----------------------------------------------
\section{Verwacht resultaat, conclusie}%
\label{sec:verwachte_resultaten}

Door het combineren van de kennis uit het literatuuronderzoek, de praktische bevindingen uit de experimenten en de inzichten uit de interviews met experts, heeft het onderzoek als doel een uitgebreid begrip te ontwikkelen van de mogelijkheden en beperkingen van Mendix in de context van de ontwikkeling van bedrijfsapplicaties. 
\\
\\
Als afsluiting van dit onderzoek wordt ook een beslissingskader opgesteld dat organisaties een gestructureerde aanpak biedt om een best passende keuze te maken tussen low-code en high-code ontwikkelmethoden. Dit kader stelt bedrijven zoals Apvine in staat om niet alleen effectiever te plannen, maar ook de risico's van late veranderingen in projecten te beperken. Hierdoor worden de ontwikkelingskosten beter beheerd en hoogwaardige oplossingen opgeleverd die aansluiten bij de unieke wensen van hun klanten.




    
    \printbibliography[heading=bibintoc]
    
\end{document}