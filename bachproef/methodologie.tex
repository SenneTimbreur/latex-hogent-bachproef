%%=============================================================================
%% Methodologie
%%=============================================================================

\chapter{\IfLanguageName{dutch}{Methodologie}{Methodology}}%
\label{ch:methodologie}

%% TODO: In dit hoofstuk geef je een korte toelichting over hoe je te werk bent
%% gegaan. Verdeel je onderzoek in grote fasen, en licht in elke fase toe wat
%% de doelstelling was, welke deliverables daar uit gekomen zijn, en welke
%% onderzoeksmethoden je daarbij toegepast hebt. Verantwoord waarom je
%% op deze manier te werk gegaan bent.
%% 
%% Voorbeelden van zulke fasen zijn: literatuurstudie, opstellen van een
%% requirements-analyse, opstellen long-list (bij vergelijkende studie),
%% selectie van geschikte tools (bij vergelijkende studie, "short-list"),
%% opzetten testopstelling/PoC, uitvoeren testen en verzamelen
%% van resultaten, analyse van resultaten, ...
%%
%% !!!!! LET OP !!!!!
%%
%% Het is uitdrukkelijk NIET de bedoeling dat je het grootste deel van de corpus
%% van je bachelorproef in dit hoofstuk verwerkt! Dit hoofdstuk is eerder een
%% kort overzicht van je plan van aanpak.
%%
%% Maak voor elke fase (behalve het literatuuronderzoek) een NIEUW HOOFDSTUK aan
%% en geef het een gepaste titel.

\section{Methodologie}%
Dit onderzoek is opgedeeld in twee fasen: eerst het in kaart brengen van het probleemdomein (het ontbreken van een beslissingskader), gevolgd door onderzoek naar het oplossingsdomein (de ontwikkeling van het kader).
\subsection{Fase 1: Analyse van het probleemdomein}
Om een grondig inzicht te krijgen in het probleemdomein worden de volgende onderzoeksmethoden gebruikt:
\subsubsection{Analyse van historische projectdata}
Om deelvragen 1 en 2 te beantwoorden ("Wat zijn de gevolgen van het ontbreken van een beslissingskader?" \; \hbox{en} \, "Welke problemen ontstaan er in projecten?"), wordt een gestructureerde analyse uitgevoerd van afgesloten projecten. Bij deze analyse worden initiële projectschattingen vergeleken met werkelijke uitkomsten om de impact op kosten en doorlooptijd nauwkeurig te kwantificeren. Daarnaast wordt de projectdocumentatie systematisch doorgenomen met als doel het identificeren van specifieke momenten waarop keuzes tussen low-code en high-code ontwikkelmethoden tot problemen hebben geleid. Het onderzoek inventariseert ook situaties waarin projectteams gedwongen waren om tijdens het project over te schakelen naar alternatieve ontwikkelmethoden vanwege onvoorziene beperkingen. Ten slotte wordt de financiële en tijdsimpact van deze late aanpassingen grondig geëvalueerd om de werkelijke kosten van suboptimale initiële technologiekeuzes inzichtelijk te maken.
\subsubsection{Interviews met experts}
Voor het beantwoorden van deelvragen 3 en 4 ("Wat zijn de huidige criteria?" \; \hbox{en} \, "Welke knelpunten ervaren projectmanagers?") worden diepte-interviews gehouden met verschillende belanghebbenden binnen het ontwikkelproces. Projectmanagers worden geïnterviewd over hun huidige besluitvormingsproces bij technologiekeuzes, waarbij specifiek wordt gefocust op de impliciete en expliciete criteria die zij hanteren. Daarnaast delen architecten hun ervaringen met technologie-keuzes, inclusief de technische overwegingen die doorslaggevend zijn bij het maken van platformkeuzes. Pre-sales consultants worden bevraagd over hun methodiek bij het inschatten van projectgeschiktheid voor verschillende ontwikkelplatformen in de offertefase. Tot slot worden ontwikkelaars geïnterviewd over de praktische uitdagingen die zij ervaren wanneer technologiekeuzes zijn gemaakt en zij deze moeten implementeren, met bijzondere aandacht voor de discrepantie tussen verwachtingen en werkelijkheid.
\subsubsection{Documentatieonderzoek}
Om deelvraag 3 verder te onderbouwen wordt bestaande interne documentatie geanalyseerd. Dit omvat een grondige bestudering van de huidige richtlijnen en procedures voor projectaanpak, waarin impliciet of expliciet keuzes worden gemaakt over ontwikkelmethoden. Ook worden pre-sales documentatie en offertes onder de loep genomen om inzicht te krijgen in de initiële afwegingen en beloftes die worden gedaan voordat een project daadwerkelijk start. Project kick-off documenten worden onderzocht om de uitgangspunten en verwachtingen aan het begin van projecten te identificeren, met specifieke aandacht voor de technologiekeuzes die in deze vroege fase worden vastgelegd. Tenslotte worden architectuurbeslissingen en design documents geanalyseerd om te begrijpen hoe technische overwegingen worden gedocumenteerd en gecommuniceerd binnen projectteams, en hoe deze documenten bijdragen aan het besluitvormingsproces rondom low-code versus high-code ontwikkeling.

\subsection{Fase 2: Onderzoek naar het oplossingsdomein}
Na het volledig in kaart brengen van het probleem, richt het onderzoek zich op het ontwikkelen van een oplossing middels:
\subsubsection{Literatuuroverzicht}
Een diepgaande analyse van bestaande documentatie over Mendix en andere low-code platforms vormt een essentieel onderdeel voor het beantwoorden van deelvraag 5 en 7. Deze analyse omvat diverse bronnen, waaronder officiële Mendix productgidsen die gedetailleerde technische specificaties en functionaliteiten beschrijven, casestudies van derden en rapporten uit de industrie die praktijkvoorbeelden en onafhankelijke evaluaties bieden, online forums en ontwikkelaarsgemeenschappen waar praktijkervaringen en knelpunten worden gedeeld, en academische publicaties over low-code ontwikkeling die theoretische onderbouwing en wetenschappelijke inzichten verschaffen over de bredere context van deze technologie.
\subsubsection{Praktisch onderzoek}
Voor de beantwoording van deelvraag 6 worden praktische tests uitgevoerd om de grenzen van het Mendix-platform te onderzoeken. Dit onderzoek richt zich op verschillende aspecten, waaronder schaalbaarheid en prestaties, waarbij de reactietijd, belastingstestresultaten en het vermogen om te schalen bij toenemende gebruikersaantallen worden geëvalueerd. Daarnaast worden de integratiemogelijkheden getest om de compatibiliteit met externe systemen, API’s en databases te beoordelen. De aanpasbaarheid en uitbreidbaarheid van het platform worden geanalyseerd door te onderzoeken in hoeverre maatwerkfunctionaliteiten en uitbreidingen kunnen worden geïmplementeerd. Ook de ontwikkelingssnelheid wordt onder de loep genomen, waarbij wordt gekeken naar de efficiëntie van de ontwikkelomgeving en de snelheid waarmee applicaties kunnen worden gebouwd en aangepast. Tot slot wordt de beveiliging en compliance beoordeeld om vast te stellen in hoeverre het platform voldoet aan relevante regelgeving en beveiligingsstandaarden. Deze tests bieden een diepgaand inzicht in de sterke en zwakke punten van Mendix en helpen bij het bepalen van de geschiktheid van het platform voor specifieke toepassingen.

\subsubsection{Reflectie op eigen ervaringen}
In dit onderzoek wordt een gestructureerde reflectie uitgevoerd op de transitie van high-code naar low-code ontwikkeling binnen een lopend Mendix-project. Hierbij worden persoonlijke ervaringen gedocumenteerd, waarbij zowel de uitdagingen als successen in de praktijk worden belicht. Er wordt een vergelijkende analyse gemaakt van de ontwikkelingsefficiëntie tussen traditionele high-code methoden en de Mendix-aanpak, met aandacht voor de verschillen in snelheid, flexibiliteit en onderhoudbaarheid. Daarnaast worden specifieke situaties geïdentificeerd waarin high-code kennis een toegevoegde waarde biedt binnen een Mendix-omgeving, bijvoorbeeld bij complexe logica of integraties met externe systemen. Tot slot wordt de leercurve geëvalueerd en worden de benodigde aanpassingen in denkwijze besproken die gepaard gaan met de overstap naar een low-code platform. Deze reflectie biedt waardevolle inzichten in de impact van low-code ontwikkeling op bestaande programmeerervaringen en werkwijzen.

\subsubsection{Evaluatie van Mendix-uitbreidingen}
In dit onderzoek wordt de effectiviteit van verschillende uitbreidingsstrategieën voor Mendix geanalyseerd. Een belangrijk aspect daarbij is de vergelijking tussen standaard Marketplace-modules en custom ontwikkeling, waarbij wordt gekeken naar factoren zoals flexibiliteit, implementatietijd en onderhoudsgemak. Daarnaast wordt de impact van Java-uitbreidingen onderzocht, met specifieke aandacht voor de invloed op onderhoudbaarheid en toekomstbestendigheid van het platform. Ook wordt een analyse uitgevoerd van de kosten-batenverhouding van verschillende uitbreidingsmethoden, waarbij zowel ontwikkel- als beheerkosten worden meegenomen. Tot slot worden de integratie-uitdagingen beoordeeld die zich voordoen bij het combineren van Mendix met externe systemen en services, om zo inzicht te krijgen in de haalbaarheid en complexiteit van verschillende uitbreidingsopties. Deze analyse draagt bij aan een beter begrip van de meest effectieve strategieën voor het uitbreiden van Mendix-toepassingen binnen verschillende bedrijfscontexten.

\subsubsection{Ontwikkeling beslissingskader}
Voor het beantwoorden van deelvraag 8, die zich richt op het destilleren van best practices en richtlijnen voor de keuze tussen Mendix en traditionele ontwikkeling, wordt een gestructureerd beslissingskader ontwikkeld. Dit kader ondersteunt projectmanagers en architecten bij het maken van weloverwogen keuzes over de inzet van Mendix in vergelijking met conventionele ontwikkelmethoden. Het biedt beslissingspunten voor de initiële keuze tussen Mendix en traditionele ontwikkeling en stelt criteria vast om te bepalen welke projectonderdelen beter geschikt zijn voor high-code ontwikkeling. Daarnaast worden richtlijnen geformuleerd voor het effectief combineren van low-code en high-code in hybride projecten, zodat beide benaderingen optimaal kunnen worden ingezet. Verder bevat het kader aanbevelingen voor de vroege identificatie van potentiële beperkingen en risico’s, waardoor projectrisico’s proactief kunnen worden gemitigeerd. Tot slot worden strategieën ontwikkeld om op basis van eerdere ervaringen en literatuur weloverwogen keuzes te maken bij toekomstige projecten. Dit beslissingskader draagt bij aan een systematische en onderbouwde aanpak voor het selecteren van de meest geschikte ontwikkelmethode binnen uiteenlopende projectcontexten.


