%%=============================================================================
%% Methodologie
%%=============================================================================

\chapter{\IfLanguageName{dutch}{Methodologie}{Methodology}}%
\label{ch:methodologie}

%% TODO: In dit hoofstuk geef je een korte toelichting over hoe je te werk bent
%% gegaan. Verdeel je onderzoek in grote fasen, en licht in elke fase toe wat
%% de doelstelling was, welke deliverables daar uit gekomen zijn, en welke
%% onderzoeksmethoden je daarbij toegepast hebt. Verantwoord waarom je
%% op deze manier te werk gegaan bent.
%% 
%% Voorbeelden van zulke fasen zijn: literatuurstudie, opstellen van een
%% requirements-analyse, opstellen long-list (bij vergelijkende studie),
%% selectie van geschikte tools (bij vergelijkende studie, "short-list"),
%% opzetten testopstelling/PoC, uitvoeren testen en verzamelen
%% van resultaten, analyse van resultaten, ...
%%
%% !!!!! LET OP !!!!!
%%
%% Het is uitdrukkelijk NIET de bedoeling dat je het grootste deel van de corpus
%% van je bachelorproef in dit hoofstuk verwerkt! Dit hoofdstuk is eerder een
%% kort overzicht van je plan van aanpak.
%%
%% Maak voor elke fase (behalve het literatuuronderzoek) een NIEUW HOOFDSTUK aan
%% en geef het een gepaste titel.

\section{Methodologie}%
Dit onderzoek is opgedeeld in twee fasen: eerst het in kaart brengen van het probleemdomein (het ontbreken van een beslissingskader), gevolgd door onderzoek naar het oplossingsdomein (de ontwikkeling van het kader).
\subsection{Fase 1: Analyse van het probleemdomein}
Om een grondig inzicht te krijgen in het probleemdomein worden de volgende onderzoeksmethoden gebruikt:
\subsubsection{Analyse van historische projectdata}
Om deelvragen 1 en 2 te beantwoorden ("Wat zijn de gevolgen van het ontbreken van een \\ beslissingskader?" \; \hbox{en} \, "Welke problemen ontstaan er in projecten?"), wordt een gestructureerde analyse uitgevoerd van afgesloten projecten. Deze analyse omvat:
\begin{itemize}
    \item Vergelijking van initiële projectschattingen versus werkelijke uitkomsten om impact op kosten en doorlooptijd te kwantificeren
    \item Analyse van projectdocumentatie om momenten te identificeren waar keuzes tussen low-code en high-code tot problemen leidden
    \item Inventarisatie van situaties waarin late overschakeling naar alternatieve ontwikkelmethoden nodig was
    \item Evaluatie van de financiële en tijdsimpact van deze late aanpassingen
\end{itemize}
\subsubsection{Interviews met experts}
Voor het beantwoorden van deelvragen 3 en 4 ("Wat zijn de huidige criteria?" \; \hbox{en} \, "Welke knelpunten ervaren projectmanagers?") worden diepte-interviews gehouden met:
\begin{itemize}
    \item Projectmanagers over hun huidige besluitvormingsproces
    \item Architecten over hun ervaringen met technologie-keuzes
    \item Pre-sales consultants over hun aanpak bij het inschatten van projectgeschiktheid
    \item Ontwikkelaars over de uitdagingen die zij ervaren bij technologie-keuzes
\end{itemize}
\subsubsection{Documentatieonderzoek}
Om deelvraag 3 verder te onderbouwen wordt bestaande interne documentatie geanalyseerd:
\begin{itemize}
    \item Huidige richtlijnen en procedures voor projectaanpak
    \item Pre-sales documentatie en offertes
    \item Project kick-off documenten
    \item Architectuurbeslissingen en design documents
\end{itemize}
\subsection{Fase 2: Onderzoek naar het oplossingsdomein}
Na het volledig in kaart brengen van het probleem, richt het onderzoek zich op het ontwikkelen van een oplossing middels:
\subsubsection{Literatuuroverzicht}
Een diepgaande analyse van bestaande documentatie over Mendix en andere low-code platforms voor het beantwoorden van deelvraag 5 en 7:
\begin{itemize}
    \item Officiële Mendix productgidsen
    \item Casestudies van derden en rapporten uit de industrie
    \item Online forums en ontwikkelaarsgemeenschappen
    \item Academische publicaties over low-code ontwikkeling
\end{itemize}
\subsubsection{Praktisch onderzoek}
Uitvoering van praktische tests voor deelvraag 6 om de grenzen van het Mendix-platform te onderzoeken:
\begin{itemize}
    \item Schaalbaarheid en prestaties
    \item Integratiemogelijkheden
    \item Aanpasbaarheid en uitbreidbaarheid
    \item Ontwikkelingssnelheid
    \item Beveiliging en compliance
\end{itemize}
\subsubsection{Ontwikkeling beslissingskader}
Voor het beantwoorden van deelvraag 8 ("Welke best practices en richtlijnen voor de keuze tussen Mendix en traditionele ontwikkeling kunnen worden gedestilleerd uit de projectervaringen en literatuur?") wordt op basis van alle verzamelde inzichten een gestructureerd beslissingskader ontwikkeld dat projectmanagers en architecten ondersteunt bij het maken van weloverwogen keuzes over de inzet van Mendix versus traditionele ontwikkelmethoden.

