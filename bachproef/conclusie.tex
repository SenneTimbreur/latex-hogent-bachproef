%%=============================================================================
%% Conclusie
%%=============================================================================

\chapter{Conclusie}%
\label{ch:conclusie}

% Trek een duidelijke conclusie, in de vorm van een antwoord op de
% onderzoeksvra(a)g(en). Wat was jouw bijdrage aan het onderzoeksdomein en
% hoe biedt dit meerwaarde aan het vakgebied/doelgroep? 
% Reflecteer kritisch over het resultaat. In Engelse teksten wordt deze sectie
% ``Discussion'' genoemd. Had je deze uitkomst verwacht? Zijn er zaken die nog
% niet duidelijk zijn?
% Heeft het onderzoek geleid tot nieuwe vragen die uitnodigen tot verder 
%onderzoek?

Deze bachelorproef had als centrale vraag: “Wat zijn de grenzen van de low-code-tool Mendix bij het ontwikkelen van zakelijke applicaties, en wanneer is high-code-ontwikkeling meer geschikt?” Aan de hand van deze hoofdvraag werden acht deelvragen onderzocht, waarvan hieronder de belangrijkste bevindingen worden samengevat.

Ten eerste leidt het ontbreken van een gestructureerd beslissingskader bij verschillen bedrijven tot een informeel keuzeproces dat sterk afhankelijk is van de ervaring en voorkeuren van individuele teams. Dit veroorzaakt onzekerheden en inconsistenties in de technologische keuzes binnen projecten, wat de effectiviteit en efficiëntie negatief beïnvloedt. Door het ontbreken van richtlijnen ontstaan er problemen zoals inefficiënte besluitvorming en beperkte mogelijkheden om tijdens projecten bij te sturen.

De huidige criteria die bijvoorbeeld \gls{FOD MOB} gebruikt bij het kiezen tussen low-code en high-code zijn vooral gebaseerd op interne expertise, de verwachte complexiteit van het project, de scope, en de beschikbaarheid van teams. Projectmanagers en architecten ervaren hierbij knelpunten doordat deze criteria weinig systematisch en expliciet zijn vastgelegd, wat het risico op suboptimale keuzes verhoogt.

Technisch gezien blijkt Mendix beperkingen te hebben bij het omgaan met complexe bedrijfslogica, intensieve real-time verwerking en integraties met legacy-systemen. In deze situaties kan een hybride oplossing of volledige high-code ontwikkeling noodzakelijk zijn. De triggers voor een overstap zijn vaak gerelateerd aan de complexiteit van de functionele eisen en technische haalbaarheid.

Daarnaast spelen projectomvang, tijdslijnen en klantvereisten een belangrijke rol bij de keuze voor low-code of high-code. Strakke deadlines en omvangrijke projecten vragen soms om de flexibiliteit van high-code oplossingen. Uit eerdere projecten bij apvine blijkt dat het zorgvuldig afwegen van deze factoren essentieel is om vertragingen en extra kosten te voorkomen.

Op basis van deze inzichten kan worden geconcludeerd dat een gestructureerd maar flexibel beslissingskader waardevol is voor zowel apvine als hun klanten. Dit kader moet niet als starre richtlijn fungeren, maar als een hulpmiddel dat keuzes motiveert en ondersteunt, rekening houdend met zowel technische als organisatorische contexten.

\bigskip

Hoewel het onderzoek duidelijk maakt dat een beslissingskader broodnodig is, roept het ook vragen op over de praktische implementatie ervan binnen bedrijven. De huidige afhankelijkheid van ervaring en ‘buikgevoel’ suggereert dat een cultuurverandering noodzakelijk is om een dergelijk kader breed te accepteren en effectief te gebruiken. Tevens is het onzeker of een uniform kader kan inspelen op de diversiteit van projecten en teams zonder aan flexibiliteit in te boeten. Het fragmenteren van technologievoorkeuren tussen teams kan een barrière vormen bij de invoering van gemeenschappelijke richtlijnen.

Verder laat het onderzoek zien dat de dynamiek van projecten, klantwensen en technologische ontwikkelingen continue aanpassingen van het kader vereisen. Dit betekent dat een beslissingskader niet een eenmalig product is, maar een levend document dat regelmatig geëvalueerd en bijgesteld moet worden. Ook is het van belang om verandermanagement en communicatie mee te nemen in het invoeringsproces, om draagvlak te creëren en weerstand te minimaliseren.

Kortom, de studie bevestigt het belang van een doordacht keuzeproces en het potentieel van een ondersteunend kader, maar benadrukt ook dat succes afhankelijk is van een goede afstemming op organisatiecultuur, flexibiliteit en continue verbetering.

