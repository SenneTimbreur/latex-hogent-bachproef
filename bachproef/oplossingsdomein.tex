%%=============================================================================
%% Methodologie
%%=============================================================================

\chapter{\IfLanguageName{dutch}{Oplossingsdomein}{Results}}%
\label{ch:resultaten}





\section{Literatuuroverzicht}
TODO
\\
\\
Een diepgaande analyse van bestaande documentatie over Mendix en andere low-code platforms vormt een essentieel onderdeel voor het beantwoorden van deelvraag 5 en 7. Deze analyse omvat diverse bronnen, waaronder officiële Mendix productgidsen die gedetailleerde technische specificaties en functionaliteiten beschrijven, casestudies van derden en rapporten uit de industrie die praktijkvoorbeelden en onafhankelijke evaluaties bieden, online forums en ontwikkelaarsgemeenschappen waar praktijkervaringen en knelpunten worden gedeeld, en academische publicaties over low-code ontwikkeling die theoretische onderbouwing en wetenschappelijke inzichten verschaffen over de bredere context van deze technologie.
\section{Praktisch onderzoek}
TODO
\\
\\
Voor de beantwoording van deelvraag 6 worden praktische tests uitgevoerd om de grenzen van het Mendix-platform te onderzoeken. Dit onderzoek richt zich op verschillende aspecten, waaronder schaalbaarheid en prestaties, waarbij de reactietijd, belastingstestresultaten en het vermogen om te schalen bij toenemende gebruikersaantallen worden geëvalueerd. Daarnaast worden de integratiemogelijkheden getest om de compatibiliteit met externe systemen, API’s en databases te beoordelen. De aanpasbaarheid en uitbreidbaarheid van het platform worden geanalyseerd door te onderzoeken in hoeverre maatwerkfunctionaliteiten en uitbreidingen kunnen worden geïmplementeerd. Ook de ontwikkelingssnelheid wordt onder de loep genomen, waarbij wordt gekeken naar de efficiëntie van de ontwikkelomgeving en de snelheid waarmee applicaties kunnen worden gebouwd en aangepast. Tot slot wordt de beveiliging en compliance beoordeeld om vast te stellen in hoeverre het platform voldoet aan relevante regelgeving en beveiligingsstandaarden. Deze tests bieden een diepgaand inzicht in de sterke en zwakke punten van Mendix en helpen bij het bepalen van de geschiktheid van het platform voor specifieke toepassingen.

\section{Reflectie op eigen ervaringen}
Op basis van mijn ervaring kan ik bevestigen dat low-code, zoals Mendix, bijzonder krachtig is voor het snel opzetten van generieke applicaties met standaardfunctionaliteiten. Het stelt je in staat om in korte tijd werkende oplossingen te bouwen, wat vooral in iteratieve of proof-of-concept contexten een grote meerwaarde biedt. Tegelijk merk ik dat zodra een project meer ‘custom’ noden heeft – zoals complexe bedrijfslogica of verfijnde integraties – de grenzen van het platform sneller voelbaar worden. In die gevallen is het een duidelijke troef om een high-code achtergrond te hebben: je begrijpt beter wat er onder de motorkap gebeurt, kunt gerichter zoeken naar workarounds en neemt bewuster beslissingen over de architectuur van je oplossing. Bovendien zie ik dat een traditionele programmeerachtergrond ook de leesbaarheid en structuur van je low-code logica ten goede komt. Je denkt in patronen, zorgt voor herbruikbaarheid en hanteert best practices die niet vanzelfsprekend zijn in een puur visuele ontwikkelomgeving.

\subsection{Reflectie op ervaringen van experts}
Er zijn ook enkele ontwikkelaars met een klassieke high-code achtergrond die inmiddels volledig actief zijn binnen low-code projecten, met name op het Mendix-platform. Ik sprak met hen over hun ervaringen en vatte hun bevindingen samen. 
Ze benadrukken dat low-code bijzonder krachtig is voor het snel ontwikkelen van generieke applicaties met standaardfunctionaliteiten. Dit maakt het ideaal voor situaties waarin snelheid en iteratieve ontwikkeling belangrijk zijn. Daarnaast wordt low-code vaak gepositioneerd als een brug tussen IT en business, doordat ook gebruikers zonder programmeerervaring relatief snel aan de slag kunnen. In de praktijk leidde dit er echter soms toe dat businessgebruikers eigen applicaties opstartten die later door ervaren ontwikkelaars moesten worden overgenomen. Deze overdracht bleek niet altijd evident: de onderliggende logica bleek vaak moeilijk leesbaar en voldeed zelden aan gangbare ontwikkelstandaarden of best practices, wat extra werk met zich meebracht om de applicatie te stabiliseren en verder te ontwikkelen.

Wat betreft de ontwikkelervaring binnen Mendix, werd er gemengd gereageerd op de version control-functionaliteit. Hoewel het systeem in principe krachtig is en goed integreert met teamwerk, kunnen foutmeldingen en merge-conflicten soms moeilijk te doorgronden zijn. Wanneer alles echter correct functioneert, biedt het versiebeheer een betrouwbare en efficiënte manier van samenwerken. De integratie van agile werkmethodieken binnen het Mendix-platform werd unaniem positief beoordeeld: user stories, sprints en voortgang kunnen rechtstreeks via de projectpagina opgevolgd en beheerd worden, wat de samenwerking tussen ontwikkelaars en stakeholders vergemakkelijkt.

Ook het gebruik van herbruikbare modules uit de Mendix Marketplace werd als een groot voordeel genoemd. Het toevoegen van bestaande componenten versnelt de ontwikkeling aanzienlijk en voorkomt dat het wiel telkens opnieuw moet worden uitgevonden. Tegelijk wordt opgemerkt dat een groot aantal van deze modules weinig tot geen documentatie bevat, waardoor het tijd kost om hun werking te doorgronden of aan te passen aan specifieke projectbehoeften. 

Over het geheel genomen beschouwen deze ontwikkelaars Mendix als een toegankelijke en efficiënte ontwikkelomgeving, die eenvoudig aanvoelt in de basis, maar bij complexere noden ook de nodige technische diepgang vereist. Hun ervaring met high-code vormt daarbij een duidelijke meerwaarde: het helpt hen om concepten sneller te begrijpen, kwalitatieve oplossingen te bouwen en de leesbaarheid en onderhoudbaarheid van hun low-code toepassingen te verbeteren.

\section{Evaluatie van Mendix-uitbreidingen}
Op basis van gesprekken met ervaren ontwikkelaars die dagelijks met Mendix werken, blijkt dat het gebruik van standaard Marketplace-modules doorgaans als efficiënt en tijdbesparend wordt ervaren, vooral bij generieke functionaliteiten. Deze modules kunnen snel geïntegreerd worden, wat de implementatietijd aanzienlijk verkort. Toch werd ook opgemerkt dat veel van deze modules onvoldoende of zelfs geheel geen documentatie bevatten. Dit gebrek aan transparantie leidt tot vertragingen tijdens implementatie en beperkt de flexibiliteit wanneer aanpassingen nodig zijn. In zulke gevallen biedt custom ontwikkeling vaak meer controle en beter afgestemde oplossingen, hoewel dit uiteraard gepaard gaat met een langere ontwikkeltijd en hogere initiële kosten.

Java-uitbreidingen binnen Mendix worden door ontwikkelaars met een high-code achtergrond beschouwd als waardevolle tools om de beperkingen van het platform te omzeilen, bijvoorbeeld bij complexe logica of integraties met externe systemen. Deze uitbreidingen verhogen echter ook de technische complexiteit van het project en kunnen de onderhoudbaarheid op lange termijn negatief beïnvloeden, zeker wanneer ze niet goed gedocumenteerd zijn of slechts door een beperkte groep binnen het team begrepen worden.

Wat betreft de kosten-batenverhouding, geven ontwikkelaars aan dat low-code in combinatie met herbruikbare modules initieel voordeliger is in zowel ontwikkeling als beheer. Naarmate de applicatie complexer wordt en meer ‘maatwerk’ vereist, verschuift dit evenwicht echter, en kunnen custom uitbreidingen of Java-integraties duurder uitvallen, zowel in termen van ontwikkeluren als bij latere aanpassingen of onderhoud.

Tot slot werd ook het integreren van externe systemen als een uitdaging benoemd. Hoewel Mendix hier voldoende ondersteuning voor biedt, vergt het combineren van low-code met externe services een goed begrip van beide kanten. Een technische achtergrond blijkt in die context bijzonder waardevol, zowel voor het begrijpen van de externe API’s als voor het bouwen van robuuste, onderhoudbare koppelingen binnen het platform. Al met al tonen deze bevindingen aan dat een doordachte strategie, met oog voor schaalbaarheid en onderhoud, essentieel is bij het uitbreiden van Mendix-toepassingen.

\section{Ontwikkeling beslissingskader}
TODO
\\
\\
Voor het beantwoorden van deelvraag 8, die zich richt op het destilleren van best practices en richtlijnen voor de keuze tussen Mendix en traditionele ontwikkeling, wordt een gestructureerd beslissingskader ontwikkeld. Dit kader ondersteunt projectmanagers en architecten bij het maken van weloverwogen keuzes over de inzet van Mendix in vergelijking met conventionele ontwikkelmethoden. Het biedt beslissingspunten voor de initiële keuze tussen Mendix en traditionele ontwikkeling en stelt criteria vast om te bepalen welke projectonderdelen beter geschikt zijn voor high-code ontwikkeling. Daarnaast worden richtlijnen geformuleerd voor het effectief combineren van low-code en high-code in hybride projecten, zodat beide benaderingen optimaal kunnen worden ingezet. Verder bevat het kader aanbevelingen voor de vroege identificatie van potentiële beperkingen en risico’s, waardoor projectrisico’s proactief kunnen worden gemitigeerd. Tot slot worden strategieën ontwikkeld om op basis van eerdere ervaringen en literatuur weloverwogen keuzes te maken bij toekomstige projecten. Dit beslissingskader draagt bij aan een systematische en onderbouwde aanpak voor het selecteren van de meest geschikte ontwikkelmethode binnen uiteenlopende projectcontexten.


