%%=============================================================================
%% Samenvatting
%%=============================================================================

% TODO: De "abstract" of samenvatting is een kernachtige (~ 1 blz. voor een
% thesis) synthese van het document.
%
% Een goede abstract biedt een kernachtig antwoord op volgende vragen:
%
% 1. Waarover gaat de bachelorproef?
% 2. Waarom heb je er over geschreven?
% 3. Hoe heb je het onderzoek uitgevoerd?
% 4. Wat waren de resultaten? Wat blijkt uit je onderzoek?
% 5. Wat betekenen je resultaten? Wat is de relevantie voor het werkveld?
%
% Daarom bestaat een abstract uit volgende componenten:
%
% - inleiding + kaderen thema
% - probleemstelling
% - (centrale) onderzoeksvraag
% - onderzoeksdoelstelling
% - methodologie
% - resultaten (beperk tot de belangrijkste, relevant voor de onderzoeksvraag)
% - conclusies, aanbevelingen, beperkingen
%
% LET OP! Een samenvatting is GEEN voorwoord!

%%---------- Nederlandse samenvatting -----------------------------------------
%
% TODO: Als je je bachelorproef in het Engels schrijft, moet je eerst een
% Nederlandse samenvatting invoegen. Haal daarvoor onderstaande code uit
% commentaar.
% Wie zijn bachelorproef in het Nederlands schrijft, kan dit negeren, de inhoud
% wordt niet in het document ingevoegd.

\IfLanguageName{english}{%
\selectlanguage{dutch}
\chapter*{Samenvatting}
\lipsum[1-4]
\selectlanguage{english}
}{}

%%---------- Samenvatting -----------------------------------------------------
% De samenvatting in de hoofdtaal van het document

\chapter{\IfLanguageName{dutch}{Samenvatting}{Abstract}}
\label{ch:samenvatting}


Deze bachelorproef onderzoekt de grenzen van het low-codeplatform Mendix bij het ontwikkelen van zakelijke applicaties, en gaat na in welke situaties high- codeontwikkeling een meer geschikte keuze vormt. De aanleiding voor dit onderzoek ontstond tijdens mijn stage, waar ik voor het eerst in aanraking kwam met Mendix en low-code in het algemeen. Via deze bachelorproef kreeg ik de kans om deze technologie niet enkel praktisch te ervaren, maar ook grondiger te analyseren binnen een bredere organisatorische en technologische context.

Low-codeontwikkeling biedt heel wat voordelen, zoals een versnelde oplevering, visuele modellering en een hogere toegankelijkheid voor gebruikers zonder technische achtergrond. In de praktijk blijkt echter dat Mendix, hoewel het platform over veel mogelijkheden beschikt, ook beperkingen kent bij projecten die complex zijn of een hoge graad van maatwerk vereisen. Vanuit deze vaststelling werd de centrale onderzoeksvraag geformuleerd: \textit{Wat zijn de grenzen van de low-code-tool Mendix bij het ontwikkelen van zakelijke applicaties, en wanneer is high-code-ontwikkeling meer geschikt?}

Om deze vraag te beantwoorden, werd een kwalitatieve onderzoeksaanpak gehanteerd. Deze bestond enerzijds uit interviews met beslissingsmakers bij \gls{FOD MOB}, een organisatie met een intern Java-team die frequent samenwerkt met low-codepartner apvine. Anderzijds werd een analyse uitgevoerd van historische projectdocumentatie om te identificeren waar de technologiekeuze tot frictie of herziening leidde. Daarnaast werd er ook praktisch onderzoek uitgevoerd, waarbij een vergelijkbare toepassing zowel in Mendix als in high-code (JavaScript/React) werd gebouwd. Deze hands-on vergelijking maakte het mogelijk om niet alleen theoretisch, maar ook functioneel inzicht te verwerven in de mogelijkheden en beperkingen van beide benaderingen.

Uit de analyse blijkt dat Mendix bijzonder krachtig is voor standaard bedrijfsprocessen, prototypes en projecten met strakke deadlines. De tool botst echter op beperkingen wanneer er nood is aan geavanceerde integraties, uitgebreide businesslogica of een hoge mate van technische controle.
Bovendien werd vastgesteld dat het gebrek aan een formeel beslissingskader binnen projectteams kan leiden tot suboptimale keuzes.

De bachelorproef concludeert dat het opstellen van een transparant en gestructureerd beslissingskader, waarin zowel technische als organisatorische factoren worden meegenomen, cruciaal is. Zo’n kader kan dienen als leidraad bij het maken van weloverwogen technologiekeuzes. Door het beslissingsproces te structureren en te onderbouwen, wordt niet alleen de efficiëntie van projecten bevorderd, maar ook de kwaliteit en het draagvlak van de gekozen oplossingen binnen IT-omgevingen versterkt.



