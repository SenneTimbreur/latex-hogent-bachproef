%%=============================================================================
%% Methodologie
%%=============================================================================

\chapter{\IfLanguageName{dutch}{Huidige gang van zaken}{Results}}%
\label{ch:gangVanZaken}

%% TODO: In dit hoofstuk geef je een korte toelichting over hoe je te werk bent
%% gegaan. Verdeel je onderzoek in grote fasen, en licht in elke fase toe wat
%% de doelstelling was, welke deliverables daar uit gekomen zijn, en welke
%% onderzoeksmethoden je daarbij toegepast hebt. Verantwoord waarom je
%% op deze manier te werk gegaan bent.
%% 
%% Voorbeelden van zulke fasen zijn: literatuurstudie, opstellen van een
%% requirements-analyse, opstellen long-list (bij vergelijkende studie),
%% selectie van geschikte tools (bij vergelijkende studie, "short-list"),
%% opzetten testopstelling/PoC, uitvoeren testen en verzamelen
%% van resultaten, analyse van resultaten, ...
%%
%% !!!!! LET OP !!!!!
%%
%% Het is uitdrukkelijk NIET de bedoeling dat je het grootste deel van de corpus
%% van je bachelorproef in dit hoofstuk verwerkt! Dit hoofdstuk is eerder een
%% kort overzicht van je plan van aanpak.
%%
%% Maak voor elke fase (behalve het literatuuronderzoek) een NIEUW HOOFDSTUK aan
%% en geef het een gepaste titel.

\section{Analyse van historische projectdata}
In de samenwerking tussen Apvine en \gls{FOD MOB}, een klant met een intern Java-developmentteam, komt de keuze tussen high-code en low-code regelmatig terug als een belangrijk discussiepunt. Het ontbreken van een duidelijk en gestructureerd beslissingskader bemoeilijkt het maken van weloverwogen technologisch onderbouwde keuzes, wat vaak leidt tot suboptimale beslissingen. Dit maakt \gls{FOD MOB} tot een uitstekend casevoorbeeld om de gevolgen van besluiteloosheid en het ontbreken van heldere richtlijnen in het technologiekeuzeproces te analyseren. In dit onderzoek worden initiële projectschattingen vergeleken met de werkelijke uitkomsten, met als doel de impact op kosten en doorlooptijd nauwkeurig te kwantificeren. Daarnaast wordt de projectdocumentatie systematisch doorgenomen om concrete momenten te identificeren waarop de keuze tussen low-code en high-code tot frictie heeft geleid. Het onderzoek inventariseert ook in welke gevallen projectteams, bijvoorbeeld binnen \gls{FOD MOB}, gedwongen waren om tijdens het project over te schakelen naar alternatieve ontwikkelmethoden, als gevolg van onvoorziene beperkingen van het gekozen platform of ontwikkelparadigma. De financiële en tijdsimpact van deze late aanpassingen wordt vervolgens grondig geëvalueerd, om de werkelijke kosten van een gebrekkig technologisch beslissingskader inzichtelijk te maken.
\\
\\
Het ontbreken van een duidelijk beslissingskader heeft aanzienlijke gevolgen voor de technologische keuzes binnen de samenwerking tussen Apvine en \gls{FOD MOB}. In plaats van op objectieve criteria en een gestructureerd evaluatieproces te vertrouwen, worden de keuzes voor high-code versus low-code vaak beïnvloed door informele voorkeuren en machtsverhoudingen binnen het team. Dit gebrek aan een formele en transparante methodologie leidt tot verschillende problemen die de effectiviteit van het project kunnen ondermijnen:
\subsubsection{Onzekerheid in de technologiekeuze}
Zonder een duidelijk beslissingskader is het voor betrokkenen moeilijk om met zekerheid te bepalen welke technologie het beste past bij de specifieke projectvereisten. Dit leidt vaak tot keuzes die niet goed afgestemd zijn op de werkelijke behoeften van het project, wat in de praktijk kan resulteren in inefficiëntie en suboptimale resultaten.
\subsubsection{Frictie binnen het projectteam}
Wanneer er geen duidelijke richtlijnen zijn voor het maken van technologische keuzes, ontstaat er vaak frictie binnen het projectteam. Teamleden hebben mogelijk verschillende opvattingen over de voor- en nadelen van low-code versus high-code, wat leidt tot onduidelijkheid en conflicten. Dit vergroot de kans op vertragingen, aangezien beslissingen niet snel genomen kunnen worden door gebrek aan consensus.
\subsubsection{Vertraagde besluitvorming}
Het proces van het maken van een keuze wordt uitgesteld of herhaald, omdat er geen objectief beslissingskader is om de discussie te structureren. Deze vertragingen kunnen het project belemmeren, vooral in de vroege fasen waarin cruciale technologische keuzes gemaakt moeten worden. Dit vertraagt de voortgang van het project en kan leiden tot hogere kosten en een langere doorlooptijd.
\subsubsection{Noodzaak tot herziening van keuzes}
Wanneer er geen formele afstemming en evaluatie plaatsvindt, worden technologische keuzes vaak pas later in het project beoordeeld. Dit kan betekenen dat gekozen technologieën niet voldoen aan de functionele eisen, schaalbaarheid of onderhoudsvereisten. Het projectteam moet dan dure en tijdrovende herzieningen doorvoeren, zoals het aanpassen van architectuur of het implementeren van alternatieve technologieën, wat het project verder vertraagt en de kosten verhoogt.
\\
\\
Kortom, het ontbreken van een helder beslissingskader maakt het lastig om weloverwogen keuzes te maken, verhoogt de kans op interne conflicten en vertragingen, en leidt uiteindelijk tot een verhoogd risico op het maken van suboptimale technologische keuzes die de voortgang en het succes van het project belemmeren.

\section{Interviews met experts}
TODO
\\
\\
Binnen deze interviews worden projectmanagers van Apvine bevraagd over hun huidige besluitvormingsproces bij het selecteren van technologieën. Apvine is een bedrijf dat zich hoofdzakelijk richt op low-code ontwikkeling, waardoor de keuze voor een low-code platform vaak als uitgangspunt wordt genomen in plaats van als open vraag. Deze commerciële en strategische voorkeur kan een significante invloed uitoefenen op de manier waarop projectmanagers technologische keuzes benaderen, en op hoe zij hun beslissingen motiveren tegenover klanten zoals \gls{FOD MOB}.

Er wordt in de interviews expliciet gefocust op zowel de expliciete als impliciete criteria die zij hanteren bij het maken van een keuze tussen low-code en high-code trajecten. In de praktijk blijkt dat deze afweging zelden volledig neutraal wordt gemaakt: projectmanagers proberen vaak de voordelen van low-code te benadrukken en mogelijke beperkingen ervan te relativeren om het platform te positioneren als een geschikte oplossing. Hierbij is het interessant om te onderzoeken in welke mate risico-inschatting, technische haalbaarheid en klantverwachtingen werkelijk een rol spelen in hun besluitvorming, of eerder achteraf worden gebruikt om een reeds gemaakte keuze te onderbouwen.

In de interviews wordt nagegaan hoe projectmanagers omgaan met de inherente voorkeur voor low-code binnen Apvine, en hoe dit hun technologische keuzes beïnvloedt. Aangezien low-code ontwikkeling een fundamenteel onderdeel vormt van de bedrijfsstrategie, is het interessant om te verkennen in hoeverre projectmanagers ruimte ervaren om alternatieven te overwegen wanneer de klantnoden of projectvereisten daar mogelijk om vragen. Daarnaast wordt onderzocht hoe zij omgaan met situaties waarin low-code minder geschikt blijkt, en of er intern ruimte bestaat om de beperkingen van het platform te erkennen of dat men voornamelijk inzet op het verdedigen van de gekozen oplossing.

Vragen voor deze interviews: 
\begin{itemize} 
    \item Welke stappen doorloopt u typisch bij het maken van een technologiekeuze voor een nieuw project? 
    \item Welke criteria weegt u het zwaarst bij de keuze tussen low-code en high-code? 
    \item In hoeverre is uw beslissing gebaseerd op objectieve data versus ervaring of intuïtie? 
    \item Worden deze criteria vastgelegd of zijn ze vooral impliciet aanwezig? 
    \item Hoe gaat u om met klanten die een voorkeur hebben voor high-code of sceptisch staan tegenover low-code? 
    \item Kunt u een voorbeeld geven van een project waarbij low-code achteraf gezien niet de beste keuze bleek? 
\end{itemize}



Daarnaast delen solution- en softwarearchitecten hun ervaringen met technologiekeuzes. Zij geven inzicht in de doorslaggevende technische overwegingen, zoals schaalbaarheid, integratiemogelijkheden en lange termijn onderhoud, die zij meenemen bij het adviseren over platformkeuzes binnen de context van \gls{FOD MOB}-projecten.

Ook pre-sales consultants van Apvine worden geïnterviewd, met de focus op hun aanpak tijdens de offerte- en analysefase. Zij lichten toe hoe zij per project inschatten welk ontwikkelplatform het meest geschikt is, welke informatie ze hiervoor nodig hebben van de klant (zoals \gls{FOD MOB}), en welke aannames hierin een rol spelen.

Tot slot worden ontwikkelaars van Apvine zelf geïnterviewd over de praktische implicaties van gemaakte technologiekeuzes. Hierbij komt vooral de kloof tussen verwachtingen en realiteit aan bod, zoals technische beperkingen die pas tijdens de implementatie duidelijk worden, of de noodzaak om onderdelen alsnog in high-code te herschrijven vanwege beperkingen van het gekozen low-code platform.


\section{Documentatieonderzoek}
TODO
\\
\\
Om deelvraag 3 verder te onderbouwen wordt bestaande interne documentatie geanalyseerd. Dit omvat een grondige bestudering van de huidige richtlijnen en procedures voor projectaanpak, waarin impliciet of expliciet keuzes worden gemaakt over ontwikkelmethoden. Ook worden pre-sales documentatie en offertes onder de loep genomen om inzicht te krijgen in de initiële afwegingen en beloftes die worden gedaan voordat een project daadwerkelijk start. Project kick-off documenten worden onderzocht om de uitgangspunten en verwachtingen aan het begin van projecten te identificeren, met specifieke aandacht voor de technologiekeuzes die in deze vroege fase worden vastgelegd. Tenslotte worden architectuurbeslissingen en design documents geanalyseerd om te begrijpen hoe technische overwegingen worden gedocumenteerd en gecommuniceerd binnen projectteams, en hoe deze documenten bijdragen aan het besluitvormingsproces rondom low-code versus high-code ontwikkeling.
\section{Analyse van kritieke use cases}
TODO
\\
\\
Om inzicht te krijgen in de beperkingen van Mendix in praktijksituaties worden specifieke use cases onderzocht waar Mendix in zijn standaardvorm tekortschoot. Dit onderzoek begint met de identificatie van projecten waarin aanvullende modules of custom ontwikkeling noodzakelijk waren om aan de projectvereisten te voldoen. Vervolgens worden de onderliggende redenen geanalyseerd waarom de standaard Mendix-functionaliteit in deze gevallen niet toereikend bleek, waarbij zowel technische als functionele beperkingen worden gedocumenteerd. De impact van deze tekortkomingen op projectdoorlooptijd en budget wordt nauwkeurig beoordeeld om de werkelijke kosten van deze beperkingen te kwantificeren. Daarnaast wordt de effectiviteit van de geïmplementeerde aanvullende oplossingen geëvalueerd om te bepalen in hoeverre deze de oorspronkelijke beperkingen succesvol hebben geadresseerd. Tot slot worden de gevonden scenario's gecategoriseerd om patronen te identificeren in situaties waarin Mendix-uitbreidingen typisch nodig zijn, wat waardevolle input vormt voor het te ontwikkelen beslissingskader.






