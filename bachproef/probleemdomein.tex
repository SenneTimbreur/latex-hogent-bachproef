%%=============================================================================
%% Methodologie
%%=============================================================================

\chapter{\IfLanguageName{dutch}{Huidige gang van zaken}{Results}}%
\label{ch:gangVanZaken}

%% TODO: In dit hoofstuk geef je een korte toelichting over hoe je te werk bent
%% gegaan. Verdeel je onderzoek in grote fasen, en licht in elke fase toe wat
%% de doelstelling was, welke deliverables daar uit gekomen zijn, en welke
%% onderzoeksmethoden je daarbij toegepast hebt. Verantwoord waarom je
%% op deze manier te werk gegaan bent.
%% 
%% Voorbeelden van zulke fasen zijn: literatuurstudie, opstellen van een
%% requirements-analyse, opstellen long-list (bij vergelijkende studie),
%% selectie van geschikte tools (bij vergelijkende studie, "short-list"),
%% opzetten testopstelling/PoC, uitvoeren testen en verzamelen
%% van resultaten, analyse van resultaten, ...
%%
%% !!!!! LET OP !!!!!
%%
%% Het is uitdrukkelijk NIET de bedoeling dat je het grootste deel van de corpus
%% van je bachelorproef in dit hoofstuk verwerkt! Dit hoofdstuk is eerder een
%% kort overzicht van je plan van aanpak.
%%
%% Maak voor elke fase (behalve het literatuuronderzoek) een NIEUW HOOFDSTUK aan
%% en geef het een gepaste titel.

\section{Analyse van historische projectdata}
In de samenwerking tussen Apvine en \gls{FOD MOB}, een klant met een intern Java-developmentteam, komt de keuze tussen high-code en low-code regelmatig terug als een belangrijk discussiepunt. Het ontbreken van een duidelijk en gestructureerd beslissingskader bemoeilijkt het maken van weloverwogen technologisch onderbouwde keuzes, wat vaak leidt tot suboptimale beslissingen. Dit maakt \gls{FOD MOB} tot een uitstekend casevoorbeeld om de gevolgen van besluiteloosheid en het ontbreken van heldere richtlijnen in het technologiekeuzeproces te analyseren. In dit onderzoek worden initiële projectschattingen vergeleken met de werkelijke uitkomsten, met als doel de impact op kosten en doorlooptijd nauwkeurig te kwantificeren. Daarnaast wordt de projectdocumentatie systematisch doorgenomen om concrete momenten te identificeren waarop de keuze tussen low-code en high-code tot frictie heeft geleid. Het onderzoek inventariseert ook in welke gevallen projectteams, bijvoorbeeld binnen \gls{FOD MOB}, gedwongen waren om tijdens het project over te schakelen naar alternatieve ontwikkelmethoden, als gevolg van onvoorziene beperkingen van het gekozen platform of ontwikkelparadigma. De financiële en tijdsimpact van deze late aanpassingen wordt vervolgens grondig geëvalueerd, om de werkelijke kosten van een gebrekkig technologisch beslissingskader inzichtelijk te maken.
\\
\\
Het ontbreken van een duidelijk beslissingskader heeft aanzienlijke gevolgen voor de technologische keuzes binnen de samenwerking tussen Apvine en \gls{FOD MOB}. In plaats van op objectieve criteria en een gestructureerd evaluatieproces te vertrouwen, worden de keuzes voor high-code versus low-code vaak beïnvloed door informele voorkeuren en machtsverhoudingen binnen het team. Dit gebrek aan een formele en transparante methodologie leidt tot verschillende problemen die de effectiviteit van het project kunnen ondermijnen:
\subsubsection{Onzekerheid in de technologiekeuze}
Zonder een duidelijk beslissingskader is het voor betrokkenen moeilijk om met zekerheid te bepalen welke technologie het beste past bij de specifieke projectvereisten. Dit leidt vaak tot keuzes die niet goed afgestemd zijn op de werkelijke behoeften van het project, wat in de praktijk kan resulteren in inefficiëntie en suboptimale resultaten.
\subsubsection{Frictie binnen het projectteam}
Wanneer er geen duidelijke richtlijnen zijn voor het maken van technologische keuzes, ontstaat er vaak frictie binnen het projectteam. Teamleden hebben mogelijk verschillende opvattingen over de voor- en nadelen van low-code versus high-code, wat leidt tot onduidelijkheid en conflicten. Dit vergroot de kans op vertragingen, aangezien beslissingen niet snel genomen kunnen worden door gebrek aan consensus.
\subsubsection{Vertraagde besluitvorming}
Het proces van het maken van een keuze wordt uitgesteld of herhaald, omdat er geen objectief beslissingskader is om de discussie te structureren. Deze vertragingen kunnen het project belemmeren, vooral in de vroege fasen waarin cruciale technologische keuzes gemaakt moeten worden. Dit vertraagt de voortgang van het project en kan leiden tot hogere kosten en een langere doorlooptijd.
\subsubsection{Noodzaak tot herziening van keuzes}
Wanneer er geen formele afstemming en evaluatie plaatsvindt, worden technologische keuzes vaak pas later in het project beoordeeld. Dit kan betekenen dat gekozen technologieën niet voldoen aan de functionele eisen, schaalbaarheid of onderhoudsvereisten. Het projectteam moet dan dure en tijdrovende herzieningen doorvoeren, zoals het aanpassen van architectuur of het implementeren van alternatieve technologieën, wat het project verder vertraagt en de kosten verhoogt.
\\
\\
Kortom, het ontbreken van een helder beslissingskader maakt het lastig om weloverwogen keuzes te maken, verhoogt de kans op interne conflicten en vertragingen, en leidt uiteindelijk tot een verhoogd risico op het maken van suboptimale technologische keuzes die de voortgang en het succes van het project belemmeren.

\section{Interviews met beslissingsmakers binnen \gls{FOD MOB}}

Binnen deze interviews worden beslissingsmakers van \gls{FOD MOB} bevraagd over hun rol in het keuzeproces omtrent technologische oplossingen in projecten waarbij zowel low-code als high-code mogelijkheden relevant zijn. \gls{FOD MOB} beschikt over een intern Java-developmentteam, wat impliceert dat er aanzienlijke technische expertise aanwezig is. Het is dan ook van belang te begrijpen hoe deze beslissingsmakers omgaan met technologische voorstellen van externe partijen, zeker wanneer deze een voorkeur uitspreken voor low-code ontwikkeling.

De interviews focussen op de formele en informele besluitvormingsprocessen binnen \gls{FOD MOB}, met bijzondere aandacht voor de criteria die bepalend zijn voor de uiteindelijke keuze van technologie. Er wordt onderzocht in welke mate beslissingen worden genomen op basis van objectieve evaluaties zoals technische haalbaarheid, onderhoudsimpact en langetermijnvisie, dan wel beïnvloed worden door andere factoren zoals strategische belangen, tijdsdruk of machtsverhoudingen binnen projectstructuren.

Daarnaast wordt verkend hoe beslissingsmakers omgaan met situaties waarin de aanvankelijk gekozen technologie gaandeweg minder geschikt blijkt te zijn. Daarbij is het van belang te achterhalen of er binnen de organisatie ruimte bestaat om bij te sturen, en welke lessen eerder genomen beslissingen hebben opgeleverd voor toekomstige projecten.

\subsection{Interviewvragen voor beslissingsmakers binnen \gls{FOD MOB}}
\begin{itemize}
    \item Kunt u beschrijven hoe binnen \gls{FOD MOB} het keuzeproces voor een technologie doorgaans verloopt bij de start van een project?
    \item Welke criteria zijn voor u doorslaggevend bij de keuze tussen low-code en high-code oplossingen?
    \item Worden deze criteria expliciet vastgelegd of zijn ze vooral impliciet aanwezig in het beslissingsproces?
    \item In hoeverre worden technologische beslissingen ondersteund door objectieve evaluaties, zoals technische haalbaarheidsstudies?
    \item Kunt u een voorbeeld geven van een project waarin een gekozen technologie (bijv. low-code) achteraf niet optimaal bleek? Hoe werd hiermee omgegaan?
    \item Hoe beoordeelt u de mate van flexibiliteit binnen \gls{FOD MOB} om in de loop van een project de technologische koers bij te stellen indien nodig?
    \item Zou een formeel en gestructureerd beslissingskader volgens u een meerwaarde kunnen bieden bij het maken van technologische keuzes binnen toekomstige projecten? Waarom wel of niet?
\end{itemize}

\subsection{Analyse van de interviews met beslissingsmakers binnen \gls{FOD MOB}}

Om inzicht te krijgen in het besluitvormingsproces rond technologische keuzes binnen \gls{FOD MOB}, werden semigestructureerde interviews afgenomen met X beslissingsmakers die betrokken zijn bij projecten waarin zowel low-code als high-code opties overwogen worden. De antwoorden werden thematisch geanalyseerd aan de hand van terugkerende patronen.

\subsubsection{Technologiekeuze bij projectstart}

Uit de interviews blijkt dat het keuzeproces voor een technologie doorgaans informeel verloopt. Hoewel er bij de start van elk project een technische analyse wordt uitgevoerd, bestaat er geen vastgelegd beslissingskader dat richting geeft aan de keuze tussen low-code en high-code. Zo gaf de geïnterviewde aan: \textit{“We evalueren vooral op basis van onze interne expertise en de complexiteit die we verwachten. Daarnaast spelen ook de scope van het project en de beschikbaarheid van onze teams een grote rol. Het is echter geen vastgelegde procedure.”} (L. Debusscher, persoonlijke communicatie, 28 mei 2025).

In de praktijk gebeurt de besluitvorming meestal via overleg tussen de development managers van zowel Mendix als Java, aangevuld met enkele softwarearchitecten. Deze groep beslist samen op basis van eerdere ervaringen, inschattingen van technische haalbaarheid en de context van het project, zonder dat dit formeel is vastgelegd of systematisch wordt geëvalueerd. Zoals de geïnterviewde toelichtte: \textit{“De keuze hangt vaak sterk af van het departement. Teams die al vertrouwd zijn met low-code en Mendix, kiezen daar meestal ook voor. Andere teams, die traditioneel met Java werken, blijven vaak bij die technologie.”} (L. Debusscher, persoonlijke communicatie, 28 mei 2025).

\subsubsection{Besluitvormingscriteria}

De geïnterviewde benadrukt dat er geen vaste besluitvormingscriteria worden gehanteerd bij de keuze voor een technologie. Hoewel factoren zoals technische haalbaarheid, integratiemogelijkheden en langetermijnonderhoud een rol spelen, berust de beslissing vooral op intuïtie en ervaring. Ook minder tastbare aspecten zoals projectdruk, politieke gevoeligheden en voorkeuren van externe partners beïnvloeden de keuze. Zoals de geïnterviewde opmerkte: \textit{“Er is wel een procedure die we volgen, maar er zijn eigenlijk geen duidelijke criteria waarop we onze keuze baseren.”} (L. Debusscher, persoonlijke communicatie, 28 mei 2025).

\subsubsection{Bijsturen tijdens het project}

In de praktijk wordt bijsturen tijdens een project zelden toegepast. Overstappen van de gekozen technologie, bijvoorbeeld van low-code naar een custom Java-oplossing, komt nauwelijks voor vanwege de hoge kosten en het aanzienlijke tijdsverlies dat hiermee gepaard gaat. Hierdoor is het vrijwel onmogelijk om halverwege het project nog van koers te veranderen zonder grote impact op planning en budget. Dit maakt de initiële besluitvorming over de te gebruiken technologie des te belangrijker, omdat de keuze in een vroeg stadium goed doordacht en zorgvuldig moet worden gemaakt om later kostbare aanpassingen te voorkomen.


\subsubsection{Nood aan een beslissingskader}

De geïnterviewde erkent het belang en het potentieel van een gestructureerd beslissingskader binnen het keuzeproces. Hoewel het resultaat van zo’n kader niet als een definitief besluit moet worden gezien, maar vooral dient als motivatie voor een gemaakte keuze, is er tegelijkertijd ook grote interesse in de uitkomsten ervan. Zoals de geïnterviewde aangeeft: \textit{“Het resultaat van het beslissingskader mag niet dienen als een effectief besluit, maar meer als motivatie voor een keuze.”} (L. Debusscher, persoonlijke communicatie, 28 mei 2025). Daarnaast benadrukt hij dat het kader voldoende flexibel moet zijn om rekening te houden met de unieke context en specifieke omstandigheden van elk project, zodat het bruikbaar blijft in diverse situaties.





\section{Documentatieonderzoek}

Uit de analyse van de interne documentatie blijkt dat de keuze tussen low-code en high-code ontwikkeling momenteel niet is verankerd in duidelijke en formeel vastgelegde richtlijnen. In plaats daarvan lijkt deze beslissing grotendeels te worden genomen op basis van intuïtie, eerdere ervaringen en de directe beschikbaarheid van zowel interne als externe ontwikkelteams. Hoewel er in documenten zoals pre-sales offertes, kick-off verslagen en architectuurbesluiten wel verwijzingen zijn naar technologiekeuzes, ontbreekt een gestructureerd afwegingskader dat systematisch wordt toegepast. Hierdoor ontstaat het risico dat keuzes niet altijd consistent of onderbouwd zijn, en dat belangrijke contextuele factoren zoals schaalbaarheid, onderhoudbaarheid of klantbehoeften, onvoldoende worden meegewogen in het besluitvormingsproces.
\\
Een bijkomende observatie is dat de mate van personalisatie van de applicatie eveneens een doorslaggevende rol speelt in de keuze voor de ontwikkelmethode. Wanneer een diep gepersonaliseerde oplossing vereist is, wordt vaker gekozen voor het interne Java-team, vanwege hun mogelijkheid om maatwerk te leveren. Daarentegen wordt bij projecten met een strakke deadline of een minder complexe functionaliteitsbehoefte vaak gekozen voor een low-code platform, omdat hiermee sneller resultaat kan worden geboekt. Deze pragmatische aanpak is begrijpelijk, maar onderstreept tegelijkertijd het ontbreken van expliciete richtlijnen die dergelijke afwegingen structureren.







