%%=============================================================================
%% Inleiding
%%=============================================================================

\chapter{\IfLanguageName{dutch}{Inleiding}{Introduction}}%
\label{ch:inleiding}

Apvine, een toonaangevend IT-consultancybedrijf, maakt intensief gebruik van low-code platformen zoals Mendix om zakelijke applicaties snel en efficiënt te ontwikkelen. Deze aanpak verlaagt de ontwikkeltijd en biedt een flexibele oplossing voor veel projecten. Echter, wanneer applicaties complexe bedrijfslogica, real-time gegevensverwerking of integraties met legacy-systemen vereisen, kunnen de beperkingen van low-code ontwikkeling zichtbaar worden.

In zulke gevallen kan een hybride of high-code aanpak nodig zijn, maar zonder een duidelijk beslissingskader is het lastig om te bepalen wanneer deze overstap gemaakt moet worden. Dit gebrek aan richtlijnen kan leiden tot projectvertragingen, hogere kosten en onverwachte aanpassingen, wat een negatieve impact heeft op zowel Apvine als haar klanten.

Dit onderzoek richt zich op het vaststellen van de grenzen van Mendix en het ontwikkelen van een gestructureerd beslissingskader, zodat projectmanagers en softwarearchitecten beter onderbouwde keuzes kunnen maken tussen low-code en high-code ontwikkeling.

\section{Kadering begrippen}
Voorafgaand aan de uitwerking van deze bachelorproef is het belangrijk om enkele kernbegrippen te kaderen. Door vooraf duidelijk te omschrijven wat we precies verstaan onder termen zoals low-code en high-code, zorgen we ervoor dat alle lezers met dezelfde basiskennis starten. Dit voorkomt verwarring en maakt de verdere analyse en besluitvorming beter te volgen en te interpreteren.
\subsection{High-code}
High-code, ook wel traditionele softwareontwikkeling genoemd, houdt in dat applicaties volledig worden geprogrammeerd met behulp van programmeertalen zoals Java, C\#, of Python. Deze aanpak biedt maximale controle, technische diepgang en flexibiliteit, en maakt het mogelijk om robuuste, schaalbare en volledig op maat gemaakte oplossingen te bouwen. High-code is bij uitstek geschikt voor toepassingen met complexe bedrijfslogica, uitgebreide integraties, hoge prestaties of specifieke beveiligingseisen. Hoewel het meer tijd en gespecialiseerde kennis vereist dan low-code, levert het doorgaans beter onderhoudbare en diep geïntegreerde software op.
\subsection{Low-code}
Low-code is een methode van softwareontwikkeling waarbij applicaties voornamelijk worden gebouwd via visuele interfaces, met gebruik van drag-and-drop componenten en vooraf geconfigureerde modules. Dit stelt gebruikers in staat om snel functionerende applicaties te creëren met minimale handmatige codering. Low-code platforms zijn ontworpen om de ontwikkelsnelheid te verhogen en softwareontwikkeling toegankelijk te maken voor een breder publiek, inclusief business users zonder diepgaande programmeerkennis. Belangrijk om te vermelden is dat deze platforms doorgaans de mogelijkheid bieden om waar nodig aanvullende code (high-code) toe te voegen. Hierdoor kunnen ontwikkelaars de standaardcomponenten uitbreiden of aanpassen aan complexe vereisten, wat de flexibiliteit van low-code aanzienlijk vergroot.


\section{\IfLanguageName{dutch}{Probleemstelling}{Problem Statement}}%
\label{sec:probleemstelling}

De groeiende populariteit van low-code platformen zoals Mendix biedt bedrijven de mogelijkheid om sneller en efficiënter zakelijke applicaties te ontwikkelen. Deze aanpak verlaagt de technische drempel en versnelt het ontwikkelproces. Toch stuiten bedrijven zoals Apvine op beperkingen van deze technologie wanneer ze complexe bedrijfslogica, real-time gegevensverwerking en integraties met legacy-systemen moeten implementeren.

Er ontbreekt momenteel een duidelijk beslissingskader binnen Apvine om te bepalen wanneer low-code een geschikte oplossing is en wanneer high-code ontwikkeling noodzakelijk wordt. Dit leidt tot projectvertragingen, hogere kosten en mogelijke klantontevredenheid door onverwachte aanpassingen in de ontwikkelingsstrategie.

Dit onderzoek richt zich op het identificeren van de grenzen van Mendix en het formuleren van een gestructureerd beslissingskader. Hiermee kunnen projectmanagers en softwarearchitecten beter onderbouwde keuzes maken over de inzet van low-code en high-code technologieën in verschillende scenario's.


\section{\IfLanguageName{dutch}{Onderzoeksvraag}{Research Question}}%
\label{sec:onderzoeksvraag}

Dit onderzoek richt zich op de onderzoeksvraag: "Wat zijn de beperkingen van de low-code-tool Mendix bij het ontwikkelen van zakelijke applicaties en welke criteria bepalen wanneer een overstap naar high-code ontwikkeling noodzakelijk wordt?" 

Om deze centrale vraag te beantwoorden, worden de volgende deelvragen onderzocht:  

\subsection{Probleemdomein}
\begin{itemize}
    \item Welke gevolgen heeft het ontbreken van een beslissingskader voor de keuze tussen low-code en high-code binnen Apvine?
    \item Welke problemen ontstaan er in projecten door het gebrek aan richtlijnen voor de keuze tussen ontwikkelmethoden?
    \item Wat zijn de huidige criteria die Apvine gebruikt bij het kiezen tussen low-code en high-code ontwikkeling?
    \item Welke knelpunten ervaren projectmanagers en architecten bij het maken van deze keuze?
\end{itemize}

\subsection{Oplossingsdomein}
\begin{itemize}
    \item Wat zijn de technische beperkingen van Mendix bij het omgaan met complexe bedrijfslogica, real-time gegevensverwerking en integraties met legacy-systemen?
    \item In welke specifieke scenario’s binnen Mendix-projecten kan een hybride of high-code oplossing nodig zijn, en welke factoren bepalen deze overstap?
    \item Hoe kunnen projectomvang, tijdslijnen en klantvereisten de beslissing beïnvloeden om wel of niet Mendix (low-code) te gebruiken of over te schakelen naar high-code?
    \item Welke lessen kunnen worden getrokken uit eerdere projecten bij Apvine waarin Mendix werd ingezet, en hoe kunnen deze inzichten bijdragen aan een effectief beslissingskader?
\end{itemize}

Door deze vragen te beantwoorden, beoogt dit onderzoek een onderbouwd beslissingskader te ontwikkelen dat projectmanagers en softwarearchitecten ondersteunt bij het maken van een weloverwogen keuze tussen low-code en high-code ontwikkelmethoden.

\section{\IfLanguageName{dutch}{Onderzoeksdoelstelling}{Research objective}}%
\label{sec:onderzoeksdoelstelling}
Het primaire doel van dit onderzoek is het ontwikkelen van een uitgebreid beslissingskader dat projectmanagers en softwarearchitecten van Apvine in staat stelt om gefundeerde keuzes te maken tussen low-code (Mendix) en high-code ontwikkelingsmethoden. Dit kader zal worden ontworpen om de technische beperkingen van Mendix te identificeren en te bepalen wanneer de overstap naar high-code oplossingen noodzakelijk wordt.
\\
\\
De concrete resultaten van dit onderzoek zullen omvatten:
\begin{itemize}
\item Een gedetailleerde analyse van de beperkingen van Mendix bij het verwerken van complexe bedrijfslogica, real-time gegevensverwerking en integraties met legacy-systemen
\item Een gestructureerd beslissingskader met duidelijke criteria voor het bepalen van de optimale ontwikkelingsaanpak
\item Een reeks praktische richtlijnen en beoordelingsinstrumenten die projectmanagers kunnen toepassen in de beginfase van projectplanning
\item Casestudies van eerdere Apvine-projecten, die succesvolle en uitdagende implementaties van low-code oplossingen illustreren
\item Aanbevelingen voor het implementeren van hybride benaderingen wanneer dit passend is
\end{itemize}


Het succes van dit onderzoek zal worden gemeten aan de hand van:
\begin{itemize}
\item De toepasbaarheid van het kader op verschillende projecttypen en klantvereisten
\item Validatie door technische experts en projectmanagers van Apvine met betrekking tot de praktische bruikbaarheid van het kader
\item Het vermogen om potentiële technische knelpunten te voorspellen voordat ze impact hebben op projectplanningen of budgetten
\item Duidelijke documentatie van het besluitvormingsproces die met klanten kan worden gedeeld om de transparantie te verbeteren
\end{itemize}


Het eindresultaat zal een uitgebreid rapport zijn met het beslissingskader, vergezeld van praktische hulpmiddelen (zoals beoordelingschecklists en beslisbomen) die direct kunnen worden geïntegreerd in de projectmethodologie van Apvine.

\section{\IfLanguageName{dutch}{Opzet van deze bachelorproef}{Structure of this bachelor thesis}}%
\label{sec:opzet-bachelorproef}

% Het is gebruikelijk aan het einde van de inleiding een overzicht te
% geven van de opbouw van de rest van de tekst. Deze sectie bevat al een aanzet
% die je kan aanvullen/aanpassen in functie van je eigen tekst.

De rest van deze bachelorproef is als volgt opgebouwd:

In Hoofdstuk~\ref{ch:stand-van-zaken} wordt een overzicht gegeven van de stand van zaken binnen het onderzoeksdomein, op basis van een literatuurstudie.

In Hoofdstuk~\ref{ch:methodologie} wordt de methodologie toegelicht en worden de gebruikte onderzoekstechnieken besproken om een antwoord te kunnen formuleren op de onderzoeksvragen.

% TODO: Vul hier aan voor je eigen hoofstukken, één of twee zinnen per hoofdstuk

In Hoofdstuk~\ref{ch:gangVanZaken} wordt een overzicht van de huidige situatie, waarbij de bestaande processen en werkwijzen worden geanalyseerd. Er wordt gekeken naar de effectiviteit van deze gang van zaken en eventuele knelpunten die zich voordoen.

In Hoofdstuk~\ref{ch:beter} worden mogelijke verbeteringen besproken, op basis van de bevindingen uit het vorige hoofdstuk. Er worden alternatieve benaderingen en oplossingen gepresenteerd die kunnen bijdragen aan een efficiëntere en effectievere werkwijze

In Hoofdstuk~\ref{ch:conclusie}, tenslotte, wordt de conclusie gegeven en een antwoord geformuleerd op de onderzoeksvragen. Daarbij wordt ook een aanzet gegeven voor toekomstig onderzoek binnen dit domein.