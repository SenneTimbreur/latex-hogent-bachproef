%===============================================================================
% LaTeX sjabloon voor de bachelorproef toegepaste informatica aan HOGENT
% Meer info op https://github.com/HoGentTIN/latex-hogent-report
%===============================================================================

\documentclass[dutch,dit,thesis]{hogentreport}

% - If necessary, replace the option `dit`' with your own department!
%   Valid entries are dbo, dbt, dgz, dit, dlo, dog, dsa, soa
% - If you write your thesis in English (remark: only possible after getting
%   explicit approval!), remove the option "dutch," or replace with "english".

\usepackage{lipsum} % For blind text, can be removed after adding actual content
\usepackage{biblatex}
\usepackage{glossaries}
\usepackage{multirow}

\makeglossaries



\newacronym{MDD}{MDD}{Model Driven Development}
\newacronym{MDA}{MDA}{Model Driven Architecture}
\newacronym{xUML}{xUML}{executable Unified Modeling Language}
\newacronym{UML}{UML}{Unified Modeling Language}
\newacronym{CASE}{CASE}{Computer Aided Software Engineering}
\newacronym{PWA}{PWA}{Progressieve Webapp}
\newacronym{UX}{UX}{User Experience (gebruikerservaring)}
\newacronym{UI}{UI}{User Interface}
\newacronym{API}{API}{Application Programming Interface}
\newacronym{AI}{AI}{Artificiële intelligentie}
\newacronym{ML}{ML}{Machine Learning}
\newacronym{MBE}{MBE}{Model-Based Engineering}
\newacronym{DEMO}{DEMO}{Design and Engineering Methodology for Organizations}
\newacronym{FOD MOB}{FOD MOB}{Federale Overheidsdienst Mobiliteit}
\newacronym{FCP}{FCP}{First Contentful Paint}
\newacronym{LCP}{LCP}{Largest Contentful Paint}
\newacronym{TBT}{TBT}{Total Blocking Time}
\newacronym{CLS}{CLS}{Cumulative Layout Shift}
\newacronym{GC}{GC}{Grabage Collection}
\newacronym{XHR}{XHR}{XMLHttpRequest}
\newacronym{DOM}{DOM}{Document Object Model}


%\usepackage{minted}
\captionsetup[table]{justification=centering, singlelinecheck=false}  % Verticaal centreren van caption bij tabellen


%% Pictures to include in the text can be put in the graphics/ folder
\graphicspath{{../graphics/}}

%% For source code highlighting, requires pygments to be installed
%% Compile with the -shell-escape flag!
%% \usepackage[chapter]{minted}
%% If you compile with the make_thesis.{bat,sh} script, use the following
%% import instead:
\usepackage{minted}  % No extra options needed
\usemintedstyle{solarized-light}  % Load the preferred style
\usepackage{float}


%% Formatting for minted environments.
\setminted{%
    autogobble,
    frame=lines,
    breaklines,
    linenos,
    tabsize=4
}


%% Ensure the list of listings is in the table of contents
\renewcommand\listoflistingscaption{%
    \IfLanguageName{dutch}{Lijst van codefragmenten}{List of listings}
}
\renewcommand\listingscaption{%
    \IfLanguageName{dutch}{Codefragment}{Listing}
}
\renewcommand*\listoflistings{%
    \cleardoublepage\phantomsection\addcontentsline{toc}{chapter}{\listoflistingscaption}%
    \listof{listing}{\listoflistingscaption}%
}


% Other packages not already included can be imported here

%%---------- Document metadata -------------------------------------------------

\author{Senne Timbreur}
\supervisor{Mevr. F. Spriet}
\cosupervisor{Mevr. J. Alexander}
\title
{Grenzen van de low-code tool Mendix}
\academicyear{\advance\year by -1 \the\year--\advance\year by 1 \the\year}
\examperiod{2}
\degreesought{\IfLanguageName{dutch}{Professionele bachelor in de toegepaste informatica}{Bachelor of applied computer science}}
\partialthesis{false} %% To display 'in partial fulfilment'
%\institution{Internshipcompany BVBA.}

%% Add global exceptions to the hyphenation here
\hyphenation{back-slash}

%% The bibliography (style and settings are  found in hogentthesis.cls)
\addbibresource{bachproef.bib}            %% Bibliography file
\addbibresource{../voorstel/voorstel.bib} %% Bibliography research proposal
\defbibheading{bibempty}{}

%% Prevent empty pages for right-handed chapter starts in twoside mode
\renewcommand{\cleardoublepage}{\clearpage}

\renewcommand{\arraystretch}{1.2}

%% Content starts here.
\begin{document}
    
    %---------- Front matter -------------------------------------------------------
    
    \frontmatter
    
    \hypersetup{pageanchor=false} %% Disable page numbering references
    %% Render a Dutch outer title page if the main language is English
    \IfLanguageName{english}{%
        %% If necessary, information can be changed here
        \degreesought{Professionele Bachelor toegepaste informatica}%
        \begin{otherlanguage}{dutch}%
            \maketitle%
        \end{otherlanguage}%
    }{}
    
    %% Generates title page content
    \maketitle
    \hypersetup{pageanchor=true}
    
    %%=============================================================================
%% Voorwoord
%%=============================================================================

\chapter{\IfLanguageName{dutch}{Woord vooraf}{Preface}}%
\label{ch:voorwoord}

Deze bachelorproef liep parallel met mijn stageperiode, waar ik voor het eerst in contact kwam met low-code ontwikkeling. Tijdens deze stage kreeg ik de kans om de technologie in de praktijk toe te passen en leerde ik de voordelen ervan kennen, zoals snelle ontwikkeling en visuele modellering. Wat begon als een nieuwe manier van werken, groeide al snel uit tot een onderwerp dat mijn interesse wekte.

Via deze bachelorproef kreeg ik de mogelijkheid om low-code niet enkel praktisch toe te passen, maar ook inhoudelijk en kritisch te onderzoeken. Ik ging dieper in op de technologische en organisatorische afwegingen die bij low-code komen kijken, en ontdekte hoe complex de besluitvorming rond technologie in realiteit kan zijn.

Ik wil graag enkele mensen bedanken die een belangrijke rol gespeeld hebben in dit proces. Bart Claeys en Jodie Alexander van apvine wil ik bedanken voor hun ondersteuning en de inzichten die ze met me deelden. Lieven Debusscher van \gls{FOD MOB} ben ik erkentelijk voor zijn bereidheid om bij te dragen aan het onderzoek en zijn waardevolle perspectieven. Tot slot gaat mijn oprechte dank uit naar mijn promotor Fien Spriet van HOGENT, voor haar begeleiding, feedback en steun gedurende dit hele traject.

Deze bachelorproef was niet enkel een academisch project, maar ook een kans om praktijk en theorie met elkaar te verbinden. Een ervaring waar ik met veel voldoening op terugkijk.



    %%=============================================================================
%% Samenvatting
%%=============================================================================

% TODO: De "abstract" of samenvatting is een kernachtige (~ 1 blz. voor een
% thesis) synthese van het document.
%
% Een goede abstract biedt een kernachtig antwoord op volgende vragen:
%
% 1. Waarover gaat de bachelorproef?
% 2. Waarom heb je er over geschreven?
% 3. Hoe heb je het onderzoek uitgevoerd?
% 4. Wat waren de resultaten? Wat blijkt uit je onderzoek?
% 5. Wat betekenen je resultaten? Wat is de relevantie voor het werkveld?
%
% Daarom bestaat een abstract uit volgende componenten:
%
% - inleiding + kaderen thema
% - probleemstelling
% - (centrale) onderzoeksvraag
% - onderzoeksdoelstelling
% - methodologie
% - resultaten (beperk tot de belangrijkste, relevant voor de onderzoeksvraag)
% - conclusies, aanbevelingen, beperkingen
%
% LET OP! Een samenvatting is GEEN voorwoord!

%%---------- Nederlandse samenvatting -----------------------------------------
%
% TODO: Als je je bachelorproef in het Engels schrijft, moet je eerst een
% Nederlandse samenvatting invoegen. Haal daarvoor onderstaande code uit
% commentaar.
% Wie zijn bachelorproef in het Nederlands schrijft, kan dit negeren, de inhoud
% wordt niet in het document ingevoegd.

\IfLanguageName{english}{%
\selectlanguage{dutch}
\chapter*{Samenvatting}
\lipsum[1-4]
\selectlanguage{english}
}{}

%%---------- Samenvatting -----------------------------------------------------
% De samenvatting in de hoofdtaal van het document

\chapter{\IfLanguageName{dutch}{Samenvatting}{Abstract}}
\label{ch:samenvatting}


Deze bachelorproef onderzoekt de grenzen van het low-codeplatform Mendix bij het ontwikkelen van zakelijke applicaties, en gaat na in welke situaties high- codeontwikkeling een meer geschikte keuze vormt. De aanleiding voor dit onderzoek ontstond tijdens mijn stage, waar ik voor het eerst in aanraking kwam met Mendix en low-code in het algemeen. Via deze bachelorproef kreeg ik de kans om deze technologie niet enkel praktisch te ervaren, maar ook grondiger te analyseren binnen een bredere organisatorische en technologische context.

Low-codeontwikkeling biedt heel wat voordelen, zoals een versnelde oplevering, visuele modellering en een hogere toegankelijkheid voor gebruikers zonder technische achtergrond. In de praktijk blijkt echter dat Mendix, hoewel het platform over veel mogelijkheden beschikt, ook beperkingen kent bij projecten die complex zijn of een hoge graad van maatwerk vereisen. Vanuit deze vaststelling werd de centrale onderzoeksvraag geformuleerd: \textit{Wat zijn de grenzen van de low-code-tool Mendix bij het ontwikkelen van zakelijke applicaties, en wanneer is high-code-ontwikkeling meer geschikt?}

Om deze vraag te beantwoorden, werd een kwalitatieve onderzoeksaanpak gehanteerd. Deze bestond enerzijds uit interviews met beslissingsmakers bij \gls{FOD MOB}, een organisatie met een intern Java-team die frequent samenwerkt met low-codepartner apvine. Anderzijds werd een analyse uitgevoerd van historische projectdocumentatie om te identificeren waar de technologiekeuze tot frictie of herziening leidde. Daarnaast werd er ook praktisch onderzoek uitgevoerd, waarbij een vergelijkbare toepassing zowel in Mendix als in high-code (JavaScript/React) werd gebouwd. Deze hands-on vergelijking maakte het mogelijk om niet alleen theoretisch, maar ook functioneel inzicht te verwerven in de mogelijkheden en beperkingen van beide benaderingen.

Uit de analyse blijkt dat Mendix bijzonder krachtig is voor standaard bedrijfsprocessen, prototypes en projecten met strakke deadlines. De tool botst echter op beperkingen wanneer er nood is aan geavanceerde integraties, uitgebreide businesslogica of een hoge mate van technische controle.
Bovendien werd vastgesteld dat het gebrek aan een formeel beslissingskader binnen projectteams kan leiden tot suboptimale keuzes.

De bachelorproef concludeert dat het opstellen van een transparant en gestructureerd beslissingskader, waarin zowel technische als organisatorische factoren worden meegenomen, cruciaal is. Zo’n kader kan dienen als leidraad bij het maken van weloverwogen technologiekeuzes. Door het beslissingsproces te structureren en te onderbouwen, wordt niet alleen de efficiëntie van projecten bevorderd, maar ook de kwaliteit en het draagvlak van de gekozen oplossingen binnen IT-omgevingen versterkt.




    
    %---------- Inhoud, lijst figuren, ... -----------------------------------------
    
    \tableofcontents
    
    % In a list of figures, the complete caption will be included. To prevent this,
    % ALWAYS add a short description in the caption!
    %
    %  \caption[short description]{elaborate description}
    %
    % If you do, only the short description will be used in the list of figures
    
    \listoffigures
    
    % If you included tables and/or source code listings, uncomment the appropriate
    % lines.
    \listoftables
    \listoflistings
    
    \cleardoublepage
    \phantomsection
    \addcontentsline{toc}{chapter}{Lijst van afkortingen}
    \printglossary[type=\acronymtype, title={Lijst van afkortingen}]
    
    
    % Als je een lijst van afkortingen of termen wil toevoegen, dan hoort die
    % hier thuis. Gebruik bijvoorbeeld de ``glossaries'' package.
    % https://www.overleaf.com/learn/latex/Glossaries
    
    %---------- Kern ---------------------------------------------------------------
    
    \mainmatter{}
    
    % De eerste hoofdstukken van een bachelorproef zijn meestal een inleiding op
    % het onderwerp, literatuurstudie en verantwoording methodologie.
    % Aarzel niet om een meer beschrijvende titel aan deze hoofdstukken te geven of
    % om bijvoorbeeld de inleiding en/of stand van zaken over meerdere hoofdstukken
    % te verspreiden!
    
    %%=============================================================================
%% Inleiding
%%=============================================================================

\chapter{\IfLanguageName{dutch}{Inleiding}{Introduction}}%
\label{ch:inleiding}

Apvine, een toonaangevend IT-consultancybedrijf, maakt intensief gebruik van low-code platformen zoals Mendix om zakelijke applicaties snel en efficiënt te ontwikkelen. Deze aanpak verlaagt de ontwikkeltijd en biedt een flexibele oplossing voor veel projecten. Echter, wanneer applicaties complexe bedrijfslogica, real-time gegevensverwerking of integraties met legacy-systemen vereisen, kunnen de beperkingen van low-code ontwikkeling zichtbaar worden.

In zulke gevallen kan een hybride of high-code aanpak nodig zijn, maar zonder een duidelijk beslissingskader is het lastig om te bepalen wanneer deze overstap gemaakt moet worden. Dit gebrek aan richtlijnen kan leiden tot projectvertragingen, hogere kosten en onverwachte aanpassingen, wat een negatieve impact heeft op zowel Apvine als haar klanten.

Dit onderzoek richt zich op het vaststellen van de grenzen van Mendix en het ontwikkelen van een gestructureerd beslissingskader, zodat projectmanagers en softwarearchitecten beter onderbouwde keuzes kunnen maken tussen low-code en high-code ontwikkeling.


\section{\IfLanguageName{dutch}{Probleemstelling}{Problem Statement}}%
\label{sec:probleemstelling}

De groeiende populariteit van low-code platformen zoals Mendix biedt bedrijven de mogelijkheid om sneller en efficiënter zakelijke applicaties te ontwikkelen. Deze aanpak verlaagt de technische drempel en versnelt het ontwikkelproces. Toch stuiten bedrijven zoals Apvine op beperkingen van deze technologie wanneer ze complexe bedrijfslogica, real-time gegevensverwerking en integraties met legacy-systemen moeten implementeren.

Er ontbreekt momenteel een duidelijk beslissingskader binnen Apvine om te bepalen wanneer low-code een geschikte oplossing is en wanneer high-code ontwikkeling noodzakelijk wordt. Dit leidt tot projectvertragingen, hogere kosten en mogelijke klantontevredenheid door onverwachte aanpassingen in de ontwikkelingsstrategie.

Dit onderzoek richt zich op het identificeren van de grenzen van Mendix en het formuleren van een gestructureerd beslissingskader. Hiermee kunnen projectmanagers en softwarearchitecten beter onderbouwde keuzes maken over de inzet van low-code en high-code technologieën in verschillende scenario's.


\section{\IfLanguageName{dutch}{Onderzoeksvraag}{Research Question}}%
\label{sec:onderzoeksvraag}

Dit onderzoek richt zich op de onderzoeksvraag: "Wat zijn de beperkingen van de low-code-tool Mendix bij het ontwikkelen van zakelijke applicaties en welke criteria bepalen wanneer een overstap naar high-code ontwikkeling noodzakelijk wordt?" 

Om deze centrale vraag te beantwoorden, worden de volgende deelvragen onderzocht:  

\subsection{Probleemdomein}
\begin{itemize}
    \item Welke gevolgen heeft het ontbreken van een beslissingskader voor de keuze tussen low-code en high-code binnen Apvine?
    \item Welke problemen ontstaan er in projecten door het gebrek aan richtlijnen voor de keuze tussen ontwikkelmethoden?
    \item Wat zijn de huidige criteria die Apvine gebruikt bij het kiezen tussen low-code en high-code ontwikkeling?
    \item Welke knelpunten ervaren projectmanagers en architecten bij het maken van deze keuze?
\end{itemize}

\subsection{Oplossingsdomein}
\begin{itemize}
    \item Wat zijn de technische beperkingen van Mendix bij het omgaan met complexe bedrijfslogica, real-time gegevensverwerking en integraties met legacy-systemen?
    \item In welke specifieke scenario’s binnen Mendix-projecten kan een hybride of high-code oplossing nodig zijn, en welke factoren bepalen deze overstap?
    \item Hoe kunnen projectomvang, tijdslijnen en klantvereisten de beslissing beïnvloeden om wel of niet Mendix (low-code) te gebruiken of over te schakelen naar high-code?
    \item Welke lessen kunnen worden getrokken uit eerdere projecten bij Apvine waarin Mendix werd ingezet, en hoe kunnen deze inzichten bijdragen aan een effectief beslissingskader?
\end{itemize}

Door deze vragen te beantwoorden, beoogt dit onderzoek een onderbouwd beslissingskader te ontwikkelen dat projectmanagers en softwarearchitecten ondersteunt bij het maken van een weloverwogen keuze tussen low-code en high-code ontwikkelmethoden.

\section{\IfLanguageName{dutch}{Onderzoeksdoelstelling}{Research objective}}%
\label{sec:onderzoeksdoelstelling}
Het primaire doel van dit onderzoek is het ontwikkelen van een uitgebreid beslissingskader dat projectmanagers en softwarearchitecten van Apvine in staat stelt om gefundeerde keuzes te maken tussen low-code (Mendix) en high-code ontwikkelingsmethoden. Dit kader zal worden ontworpen om de technische beperkingen van Mendix te identificeren en te bepalen wanneer de overstap naar high-code oplossingen noodzakelijk wordt.
\\
\\
De concrete resultaten van dit onderzoek zullen omvatten:
\begin{itemize}
\item Een gedetailleerde analyse van de beperkingen van Mendix bij het verwerken van complexe bedrijfslogica, real-time gegevensverwerking en integraties met legacy-systemen
\item Een gestructureerd beslissingskader met duidelijke criteria voor het bepalen van de optimale ontwikkelingsaanpak
\item Een reeks praktische richtlijnen en beoordelingsinstrumenten die projectmanagers kunnen toepassen in de beginfase van projectplanning
\item Casestudies van eerdere Apvine-projecten, die succesvolle en uitdagende implementaties van low-code oplossingen illustreren
\item Aanbevelingen voor het implementeren van hybride benaderingen wanneer dit passend is
\end{itemize}


Het succes van dit onderzoek zal worden gemeten aan de hand van:
\begin{itemize}
\item De toepasbaarheid van het kader op verschillende projecttypen en klantvereisten
\item Validatie door technische experts en projectmanagers van Apvine met betrekking tot de praktische bruikbaarheid van het kader
\item Het vermogen om potentiële technische knelpunten te voorspellen voordat ze impact hebben op projectplanningen of budgetten
\item Duidelijke documentatie van het besluitvormingsproces die met klanten kan worden gedeeld om de transparantie te verbeteren
\end{itemize}


Het eindresultaat zal een uitgebreid rapport zijn met het beslissingskader, vergezeld van praktische hulpmiddelen (zoals beoordelingschecklists en beslisbomen) die direct kunnen worden geïntegreerd in de projectmethodologie van Apvine.

\section{\IfLanguageName{dutch}{Opzet van deze bachelorproef}{Structure of this bachelor thesis}}%
\label{sec:opzet-bachelorproef}

% Het is gebruikelijk aan het einde van de inleiding een overzicht te
% geven van de opbouw van de rest van de tekst. Deze sectie bevat al een aanzet
% die je kan aanvullen/aanpassen in functie van je eigen tekst.

De rest van deze bachelorproef is als volgt opgebouwd:

In Hoofdstuk~\ref{ch:stand-van-zaken} wordt een overzicht gegeven van de stand van zaken binnen het onderzoeksdomein, op basis van een literatuurstudie.

In Hoofdstuk~\ref{ch:methodologie} wordt de methodologie toegelicht en worden de gebruikte onderzoekstechnieken besproken om een antwoord te kunnen formuleren op de onderzoeksvragen.

% TODO: Vul hier aan voor je eigen hoofstukken, één of twee zinnen per hoofdstuk

In Hoofdstuk~\ref{ch:conclusie}, tenslotte, wordt de conclusie gegeven en een antwoord geformuleerd op de onderzoeksvragen. Daarbij wordt ook een aanzet gegeven voor toekomstig onderzoek binnen dit domein.
    \chapter{\IfLanguageName{dutch}{Stand van zaken}{State of the art}}%
\label{ch:stand-van-zaken}

\section{Introductie}
Mendix vertegenwoordigt een van de toonaangevende low-code ontwikkelplatformen in het huidige softwareontwikkelingslandschap. Als implementatie van Model-Driven Development (MDD) principes stelt Mendix zowel professionele ontwikkelaars als zakelijke gebruikers in staat om applicaties te creëren via visuele modellering in plaats van traditionele codering. Het platform heeft aanzienlijke tractie gewonnen onder ondernemingen die hun digitale transformatie-initiatieven willen versnellen door ontwikkeltijd en technische complexiteit te verminderen.
\\
Hoewel Mendix via zijn modelgestuurde aanpak tal van voordelen biedt, waaronder verhoogde ontwikkelingssnelheid en bredere participatie van niet-technische belanghebbenden, presenteert het ook bepaalde beperkingen die de effectiviteit in verschillende contexten beïnvloeden. Deze thesis onderzoekt Mendix als een hedendaagse manifestatie van Model-Driven Development en analyseert kritisch de beperkingen ervan op het gebied van technische mogelijkheden, uitdagingen bij organisatorische adoptie en comparatieve nadelen ten opzichte van vergelijkbare platformen.
\\
Door zowel de theoretische onderbouwing van MDD als de praktische implementatie ervan in Mendix te begrijpen, beoogt dit onderzoek een uitgebreide beoordeling te geven van waar en hoe Mendix mogelijk tekortschiet in het aanpakken van complexe enterprise software behoeften, ondanks de innovatieve benadering van applicatieontwikkeling.
\newpage

\section{\IfLanguageName{dutch}{Model-Driven Development}{Model-Driven Development}}%
In dit hoofdstuk wordt de theoretische achtergrond van Model-Driven Development (MDD) en low-code ontwikkeling uiteengezet, met een specifieke focus op Mendix. Dit sluit aan bij de introductie van de stand van zaken, waarin het belang van MDD en low-code ontwikkeling werd geïntroduceerd. Dit hoofdstuk biedt een diepgaande analyse van MDD, inclusief de voordelen en beperkingen, en gaat na hoe Mendix binnen dit kader past.

\subsection{\IfLanguageName{dutch}{Kernconcepten van Model-Driven Development}{Core Concepts of Model-Driven Development}}%

Volgens het onderzoek \textcite{Hailpern2006} is MDD “een software engineering aanpak die bestaat uit de toepassing van modellen om het abstractieniveau te verhogen” in softwareontwikkeling. Deze aanpak ontstond als een natuurlijke evolutie in de overgang van low-level programmeertalen zoals assembly naar higher-level talen zoals Java en C\#, waarbij modellen de volgende stap in deze abstractie-evolutie vertegenwoordigen.

Het fundamentele principe achter MDD is dat het werken op hogere abstractieniveaus ontwikkelaars in staat stelt om complexe systemen effectiever en met minder inspanning te beheren. Dit komt overeen met de historische evolutie van programmeertalen, waarbij elke nieuwe generatie de abstractie vergrootte om de productiviteit te verbeteren en de cognitieve belasting van ontwikkelaars te verminderen.


\subsection{\IfLanguageName{dutch}{Historische context en evolutie}{Historical Context and Evolution}}%
De wortels van MDD gaan meer dan 20 jaar terug en hebben zich in verschillende belangrijke fasen ontwikkeld. In de jaren 80 kwamen de eerste ambitieuze modelleermethodologieën op om vereisten voor informatiesystemen vast te leggen. Dit werd gevolgd door het tijdperk van de Computer-Aided Software Engineering (CASE) tools in de jaren 1990, die probeerden om informatiesystemen gedeeltelijk te genereren vanuit modellen, hoewel veel CASE's alleen codesjablonen konden genereren die handmatig moesten worden ingevuld. Hedendaagse benaderingen zijn onder andere OMG's Model Driven Architecture (MDA) en uitvoerbare UML (xUML), die meer verfijnde implementaties van modelgedreven principes vertegenwoordigen.

Het MDD-landschap omvat verschillende benaderingen met onderscheidende kenmerken. OMG's Model Driven Architecture richt zich op meerdere abstractieniveaus van modellen en transformaties tussen deze niveaus. MDA maakt meestal gebruik van Computation Independent Models (CIMs) die business requirements representeren, Platform Independent Models (PIMs) die systeemfunctionaliteit specificeren zonder technische details, en Platform Specific Models (PSMs) die platformspecifieke implementatiedetails bevatten.

Executable UML (xUML) daarentegen is gericht op het maken van volledig uitvoerbare modellen die 100\% van de benodigde code kunnen genereren, zodat handmatig coderen niet meer nodig is. Eerdere generaties MDD omvatten op formulieren gebaseerde CASE-tools zoals Oracle Forms en Microsoft Access Forms, die beperkte controle boden over het genereren van code, maar toegankelijke interfaces boden voor snelle applicatieontwikkeling. Moderne MDD-platforms zoals OptimalJ en Mendix vertegenwoordigen de hedendaagse aanpak en bieden verschillende niveaus van controle over modeltransformatie en codegeneratie.

\subsection{\IfLanguageName{dutch}{Voordelen van Model-Driven Development}{Benefits of Model-Driven Development}}%
De literatuur identificeert verschillende belangrijke voordelen van het gebruik van MDD-benaderingen. Door te werken met goed gedefinieerde modellen kunnen ontwikkelaars het aantal implementatiefouten verminderen, waardoor de softwarekwaliteit verbetert. Hogere abstractieniveaus stellen ontwikkelaars in staat om zich te richten op bedrijfslogica in plaats van technische implementatiedetails, wat de productiviteit van ontwikkelaars verhoogt. Modellen geven een duidelijker inzicht in de systeemstructuur en het gedrag, waardoor het ontwikkelproces beter onder controle is.

Bovendien vermindert het automatisch genereren van code de handmatige codeerinspanning en de bijbehorende fouten, wat leidt tot lagere ontwikkel- en onderhoudskosten. Snellere ontwikkelcycli helpen organisaties efficiënter om te gaan met hun applicatieontwikkelingsbehoeften, waardoor de ontwikkelingsachterstand afneemt. De mogelijkheid om snel iteraties uit te voeren en werkende functionaliteit te demonstreren verbetert de betrokkenheid van belanghebbenden, waardoor de klanttevredenheid toeneemt.

\subsection{\IfLanguageName{dutch}{Belangrijkste functionaliteitsgebieden voor MDD-tools}{Benefits of Model-Driven Development}}%
Om een MDD-tool effectief te laten zijn, moet het verschillende kritieke functionaliteiten ondersteunen, die kunnen worden gecategoriseerd in ondersteuning van zowel het modelleren als het ontwikkelproces.

Effectieve ondersteuning bij het modelleren vereist dat tools de juiste abstractieniveaus hanteren, waarbij irrelevante details worden verborgen terwijl essentiële concepten worden blootgelegd. 

Dit is vooral belangrijk omdat belanghebbenden modellen voor verschillende doeleinden gebruiken. Modellen moeten begrijpelijk zijn voor zowel technische als niet-technische belanghebbenden, idealiter met behulp van intuïtieve, voorspelbare notatie - veel tools maken om deze reden gebruik van UML, omdat dit wordt beschouwd als de de-facto standaard in de industrie \autocite{Marin2015} .Modellen moeten uitvoerbaar zijn, zelfs als ze incompleet zijn, zodat incrementele ontwikkeling mogelijk is en ontwikkelaars het gedrag van het systeem kunnen voorspellen door middel van experimenten of formele analyse. 

Volwassen MDD-tools ondersteunen de verfijning van modellen en transformaties tussen verschillende abstractieniveaus, waardoor aanpasbare model-naar-model en model-naar-code transformaties mogelijk zijn. Volgens \textcite{Marin2015} “moeten MDD-tools de uitvoering van modellen mogelijk maken, ook al zijn ze onvolledig, maar wel geldig”, wat modelcorrectie en -validatie in een vroeg stadium vergemakkelijkt.

Ondersteuning van het ontwikkelproces omvat een reeks functies. Tools moeten duidelijke feedback geven over fouten, idealiter door direct te wijzen naar de modelcomponenten die problemen veroorzaken, vergelijkbaar met hoe compilers problematische code markeren. 
Omdat software meestal door teams wordt ontwikkeld, moeten MDD-tools modelvergelijking, samenvoeging en versiebeheer ondersteunen. 

\textcite{Marin2015} benadrukken dat “versiebeheer van modellen absoluut noodzakelijk is om industriële samenwerkingsprojecten te beheren, waarbij verschillende leden van een ontwikkelteam aan hetzelfde model kunnen werken”.Effectieve tools moeten snel compileren en implementeren, met een bijzonder efficiënte afhandeling van incrementele wijzigingen.


MDD-tools moeten integreren met bestaande systemen en ontwikkelomgevingen, waardoor verbindingen met ERP-systemen, legacy applicaties en andere bedrijfsinfrastructuur mogelijk worden. Een van de belangrijkste voordelen van MDD is dat een breder scala aan mensen kan deelnemen aan de ontwikkeling, waaronder bedrijfskundigen met beperkte programmeervaardigheden. Tools moeten de definitie en het gebruik van herbruikbare componenten, patronen en best practices in projecten vergemakkelijken. 

Door deze functies op te nemen kunnen MDD-tools beter aansluiten op de behoeften van de industrie en de succesvolle toepassing van het MDD-paradigma in softwareontwikkelingsprojecten ondersteunen.

\subsection{\IfLanguageName{dutch}{De evolutie naar low-code platformen}{The Evolution Toward Low-Code Platforms}}%
Model-Driven Development (MDD) is de laatste jaren sterk geëvolueerd, vooral met de opkomst van low-code platformen zoals Mendix. Deze platforms vertegenwoordigen een verfijning van MDD-principes, waardoor ze toegankelijker worden voor ontwikkelaars met verschillende vaardigheidsniveaus, terwijl de kernvoordelen van modelgedreven benaderingen behouden blijven.

Zoals \textcite{Henkel2010} aantoonden in hun analyse van Mendix, proberen moderne MDD-tools een balans te vinden tussen abstractie en controle, waardoor ontwikkelaars op bedrijfsniveau kunnen redeneren terwijl ze toch voldoende controle houden over het systeemgedrag. Deze evolutie heeft MDD-principes toegankelijk gemaakt voor een breder publiek, waardoor softwareontwikkeling democratischer wordt dan alleen voor traditionele programmeerexperts. \textcite{Henkel2010} benadrukken dat “low-code platforms zoals Mendix met succes de kloof hebben overbrugd tussen abstractie op hoog niveau en technische implementatie, waardoor zowel zakelijke belanghebbenden als ontwikkelaars effectief kunnen samenwerken.”

Concluderend kan gesteld worden dat Model-Driven Development een significante vooruitgang betekent in software engineering methodologie, door modellen te verheffen van secundaire documentatie tot primaire ontwikkelartefacten. Door het abstractieniveau te verhogen stelt MDD ontwikkelaars in staat om complexiteit effectiever te managen, waardoor de productiviteit kan toenemen, de kwaliteit kan verbeteren en softwareontwikkeling toegankelijker wordt voor een breder scala aan belanghebbenden.De integratie van MDD-principes in low-code platforms heeft het bereik verder vergroot, waardoor het een krachtig hulpmiddel is geworden voor moderne softwareontwikkeling.

\section{\IfLanguageName{dutch}{Low-code platformen}{Low-code platforms}}%
In de volgende paragrafen zullen we de belangrijkste kenmerken, sterke punten en beperkingen van verschillende toonaangevende low-code ontwikkelplatformen onderzoeken en vergelijken, waaronder OutSystems, Joget DX en Mendix. Elk platform biedt unieke mogelijkheden en komt tegemoet aan verschillende organisatorische behoeften en use cases. Door hun benaderingen van visuele ontwikkeling, schaalbaarheid, inzetmogelijkheden, maatwerk en integratiemogelijkheden te onderzoeken, willen we een duidelijk inzicht geven in hoe deze platforms zich tot elkaar verhouden. Uiteindelijk zal deze vergelijking duidelijk maken waarom Mendix de meest uitgebreide en veelzijdige oplossing is, met de beste balans tussen flexibiliteit, schaalbaarheid en geavanceerde functies voor bedrijven die hun digitale transformatie willen versnellen.
\subsection{\IfLanguageName{dutch}{OutSystems}{OutSystems}}
OutSystems is een robuust low-code platform dat bekend staat om zijn visuele ontwikkelomgeving en mogelijkheden op enterprise-niveau. Het blinkt uit in snelle applicatieontwikkeling en biedt zowel cloud-native als on-premises implementatieopties, die tegemoet komen aan uiteenlopende organisatorische behoeften \autocite{Sido2024}. OutSystems kan ook bogen op een levendig ecosysteem en community, samen met sterke beveiligingsfuncties en naleving van industriestandaarden. Het licentiemodel en de kostenbeperkingen kunnen echter onbetaalbaar zijn voor sommige organisaties, en de beperkte database-integratie en migratie-uitdagingen kunnen flexibiliteit in de weg staan \autocite{Sido2024}. Daarnaast kunnen de kwaliteit van de documentatie en de geïsoleerde ondersteuning van de community barrières opwerpen voor ontwikkelaars die het potentieel van het platform willen maximaliseren.
\subsection{\IfLanguageName{dutch}{Joget DX}{Joget DX}}
Joget DX daarentegen is een open-source platform dat de nadruk legt op eenvoud en snelle applicatieontwikkeling. Het is vooral sterk in de ontwikkeling van progressieve webapps (PWA's), gebruikerservaring (UX) en integratie met DevOps-praktijken \autocite{Sido2024}. Joget DX biedt ook uitbreidbaarheid via add-on builders en verbeterde workflowmogelijkheden, waardoor het een flexibele keuze is voor bedrijven die processen willen automatiseren. De beperkingen van het abonnementsmodel, de beperkte databasetoegang en de app-beperkingen kunnen de schaalbaarheid en aanpasbaarheid echter beperken \autocite{Sido2024}. Bovendien kunnen de leercurve en aanpassingscomplexiteit van het platform uitdagingen vormen voor gebruikers die overstappen van traditionele ontwikkelmethoden.
\subsection{\IfLanguageName{dutch}{Mendix}{Mendix}}
Mendix onderscheidt zich als het meest uitgebreide en veelzijdige low-code platform en biedt een breed scala aan functies voor zowel technische als niet-technische gebruikers. De modelgedreven ontwikkelaanpak, gecombineerd met microservices en containerisatie, zorgt voor schaalbaarheid, flexibiliteit en draagbaarheid. Mendix ondersteunt zowel cloud-native als on-premise infrastructuren en biedt daarmee flexibiliteit in de inzetmogelijkheden. Het volledige levenscyclusbeheer van het platform, de mogelijkheden voor kunstmatige intelligentie (AI) en machine learning (ML) en de functies voor procesautomatisering maken het een krachtig hulpmiddel voor ondernemingen \autocite{Sido2024}. Bovendien zorgen de openheid en uitbreidbaarheid van Mendix voor een naadloze integratie met externe systemen en aanpassingen om aan specifieke bedrijfsbehoeften te voldoen. Hoewel Mendix een aantal beperkingen heeft, zoals beperkte aanpasbaarheid van thema's en potentiële vendor lock-in, wegen de algehele mogelijkheden en het gebruiksgemak op tegen deze nadelen \autocite{Sido2024}.
\subsection{\IfLanguageName{dutch}{Keuze voor Mendix als low-code platform}{Choosing Mendix as low-code platform}}
Hoewel OutSystems en Joget DX waardevolle functies en mogelijkheden bieden, komt Mendix naar voren als het beste platform vanwege zijn uitgebreide en flexibele benadering van low-code ontwikkeling. Het vermogen om complexe bedrijfsapplicaties te ondersteunen, gecombineerd met de schaalbaarheid, uitbreidbaarheid en sterke focus op gebruikerservaring, maakt het de ideale keuze voor organisaties die op efficiënte en effectieve wijze digitale transformatie willen stimuleren.

\section{Mendix vs High-code}
\section{Mendix die te kort komt en high-code die aanvult}

    %%=============================================================================
%% Methodologie
%%=============================================================================

\chapter{\IfLanguageName{dutch}{Methodologie}{Methodology}}%
\label{ch:methodologie}

%% TODO: In dit hoofstuk geef je een korte toelichting over hoe je te werk bent
%% gegaan. Verdeel je onderzoek in grote fasen, en licht in elke fase toe wat
%% de doelstelling was, welke deliverables daar uit gekomen zijn, en welke
%% onderzoeksmethoden je daarbij toegepast hebt. Verantwoord waarom je
%% op deze manier te werk gegaan bent.
%% 
%% Voorbeelden van zulke fasen zijn: literatuurstudie, opstellen van een
%% requirements-analyse, opstellen long-list (bij vergelijkende studie),
%% selectie van geschikte tools (bij vergelijkende studie, "short-list"),
%% opzetten testopstelling/PoC, uitvoeren testen en verzamelen
%% van resultaten, analyse van resultaten, ...
%%
%% !!!!! LET OP !!!!!
%%
%% Het is uitdrukkelijk NIET de bedoeling dat je het grootste deel van de corpus
%% van je bachelorproef in dit hoofstuk verwerkt! Dit hoofdstuk is eerder een
%% kort overzicht van je plan van aanpak.
%%
%% Maak voor elke fase (behalve het literatuuronderzoek) een NIEUW HOOFDSTUK aan
%% en geef het een gepaste titel.

Dit onderzoek is opgedeeld in twee fasen: eerst het in kaart brengen van het probleemdomein (het ontbreken van een beslissingskader), gevolgd door onderzoek naar het oplossingsdomein (de ontwikkeling van het kader).
\section{Fase 1: Analyse van het probleemdomein}
Om een grondig inzicht te krijgen in het probleemdomein worden de volgende onderzoeksmethoden gebruikt:
\subsection{Analyse van historische projectdata}
Om deelvragen 1 en 2 te beantwoorden ("Wat zijn de gevolgen van het ontbreken van een beslissingskader?" \; \hbox{en} \, "Welke problemen ontstaan er in projecten?"), wordt een gestructureerde analyse uitgevoerd van afgesloten projecten. Bij deze analyse worden initiële projectschattingen vergeleken met werkelijke uitkomsten om de impact op kosten en doorlooptijd nauwkeurig te kwantificeren. Daarnaast wordt de projectdocumentatie systematisch doorgenomen met als doel het identificeren van specifieke momenten waarop keuzes tussen low-code en high-code ontwikkelmethoden tot problemen hebben geleid. Het onderzoek inventariseert ook situaties waarin projectteams gedwongen waren om tijdens het project over te schakelen naar alternatieve ontwikkelmethoden vanwege onvoorziene beperkingen. Ten slotte wordt de financiële en tijdsimpact van deze late aanpassingen grondig geëvalueerd om de werkelijke kosten van suboptimale initiële technologiekeuzes inzichtelijk te maken.
\subsection{Interviews met experts}
Voor het beantwoorden van deelvragen 3 en 4 ("Wat zijn de huidige criteria?" \; \hbox{en} \, "Welke knelpunten ervaren projectmanagers?") worden diepte-interviews gehouden met verschillende belanghebbenden binnen het ontwikkelproces. Projectmanagers worden geïnterviewd over hun huidige besluitvormingsproces bij technologiekeuzes, waarbij specifiek wordt gefocust op de impliciete en expliciete criteria die zij hanteren. Daarnaast delen architecten hun ervaringen met technologie-keuzes, inclusief de technische overwegingen die doorslaggevend zijn bij het maken van platformkeuzes. Pre-sales consultants worden bevraagd over hun methodiek bij het inschatten van projectgeschiktheid voor verschillende ontwikkelplatformen in de offertefase. Tot slot worden ontwikkelaars geïnterviewd over de praktische uitdagingen die zij ervaren wanneer technologiekeuzes zijn gemaakt en zij deze moeten implementeren, met bijzondere aandacht voor de discrepantie tussen verwachtingen en werkelijkheid.
\subsection{Documentatieonderzoek}
Om deelvraag 3 verder te onderbouwen wordt bestaande interne documentatie geanalyseerd. Dit omvat een grondige bestudering van de huidige richtlijnen en procedures voor projectaanpak, waarin impliciet of expliciet keuzes worden gemaakt over ontwikkelmethoden. Ook worden pre-sales documentatie en offertes onder de loep genomen om inzicht te krijgen in de initiële afwegingen en beloftes die worden gedaan voordat een project daadwerkelijk start. Project kick-off documenten worden onderzocht om de uitgangspunten en verwachtingen aan het begin van projecten te identificeren, met specifieke aandacht voor de technologiekeuzes die in deze vroege fase worden vastgelegd. Tenslotte worden architectuurbeslissingen en design documents geanalyseerd om te begrijpen hoe technische overwegingen worden gedocumenteerd en gecommuniceerd binnen projectteams, en hoe deze documenten bijdragen aan het besluitvormingsproces rondom low-code versus high-code ontwikkeling.

\section{Fase 2: Onderzoek naar het oplossingsdomein}
Na het volledig in kaart brengen van het probleem, richt het onderzoek zich op het ontwikkelen van een oplossing middels:
\subsection{Literatuuroverzicht}
Een diepgaande analyse van bestaande documentatie over Mendix en andere low-code platforms vormt een essentieel onderdeel voor het beantwoorden van deelvraag 5 en 7. Deze analyse omvat diverse bronnen, waaronder officiële Mendix productgidsen die gedetailleerde technische specificaties en functionaliteiten beschrijven, casestudies van derden en rapporten uit de industrie die praktijkvoorbeelden en onafhankelijke evaluaties bieden, online forums en ontwikkelaarsgemeenschappen waar praktijkervaringen en knelpunten worden gedeeld, en academische publicaties over low-code ontwikkeling die theoretische onderbouwing en wetenschappelijke inzichten verschaffen over de bredere context van deze technologie.
\subsection{Praktisch onderzoek}
Voor de beantwoording van deelvraag 6 worden praktische tests uitgevoerd om de grenzen van het Mendix-platform te onderzoeken. Dit onderzoek richt zich op verschillende aspecten, waaronder schaalbaarheid en prestaties, waarbij de reactietijd, belastingstestresultaten en het vermogen om te schalen bij toenemende gebruikersaantallen worden geëvalueerd. Daarnaast worden de integratiemogelijkheden getest om de compatibiliteit met externe systemen, API’s en databases te beoordelen. De aanpasbaarheid en uitbreidbaarheid van het platform worden geanalyseerd door te onderzoeken in hoeverre maatwerkfunctionaliteiten en uitbreidingen kunnen worden geïmplementeerd. Ook de ontwikkelingssnelheid wordt onder de loep genomen, waarbij wordt gekeken naar de efficiëntie van de ontwikkelomgeving en de snelheid waarmee applicaties kunnen worden gebouwd en aangepast. Tot slot wordt de beveiliging en compliance beoordeeld om vast te stellen in hoeverre het platform voldoet aan relevante regelgeving en beveiligingsstandaarden. Deze tests bieden een diepgaand inzicht in de sterke en zwakke punten van Mendix en helpen bij het bepalen van de geschiktheid van het platform voor specifieke toepassingen.

\subsection{Reflectie op eigen ervaringen}
In dit onderzoek wordt een gestructureerde reflectie uitgevoerd op de transitie van high-code naar low-code ontwikkeling binnen een lopend Mendix-project. Hierbij worden persoonlijke ervaringen gedocumenteerd, waarbij zowel de uitdagingen als successen in de praktijk worden belicht. Er wordt een vergelijkende analyse gemaakt van de ontwikkelingsefficiëntie tussen traditionele high-code methoden en de Mendix-aanpak, met aandacht voor de verschillen in snelheid, flexibiliteit en onderhoudbaarheid. Daarnaast worden specifieke situaties geïdentificeerd waarin high-code kennis een toegevoegde waarde biedt binnen een Mendix-omgeving, bijvoorbeeld bij complexe logica of integraties met externe systemen. Tot slot wordt de leercurve geëvalueerd en worden de benodigde aanpassingen in denkwijze besproken die gepaard gaan met de overstap naar een low-code platform. Deze reflectie biedt waardevolle inzichten in de impact van low-code ontwikkeling op bestaande programmeerervaringen en werkwijzen.

\subsection{Evaluatie van Mendix-uitbreidingen}
In dit onderzoek wordt de effectiviteit van verschillende uitbreidingsstrategieën voor Mendix geanalyseerd. Een belangrijk aspect daarbij is de vergelijking tussen standaard Marketplace-modules en custom ontwikkeling, waarbij wordt gekeken naar factoren zoals flexibiliteit, implementatietijd en onderhoudsgemak. Daarnaast wordt de impact van Java-uitbreidingen onderzocht, met specifieke aandacht voor de invloed op onderhoudbaarheid en toekomstbestendigheid van het platform. Ook wordt een analyse uitgevoerd van de kosten-batenverhouding van verschillende uitbreidingsmethoden, waarbij zowel ontwikkel- als beheerkosten worden meegenomen. Tot slot worden de integratie-uitdagingen beoordeeld die zich voordoen bij het combineren van Mendix met externe systemen en services, om zo inzicht te krijgen in de haalbaarheid en complexiteit van verschillende uitbreidingsopties. Deze analyse draagt bij aan een beter begrip van de meest effectieve strategieën voor het uitbreiden van Mendix-toepassingen binnen verschillende bedrijfscontexten.

\subsection{Ontwikkeling beslissingskader}
Voor het beantwoorden van deelvraag 8, die zich richt op het destilleren van best practices en richtlijnen voor de keuze tussen Mendix en traditionele ontwikkeling, wordt een gestructureerd beslissingskader ontwikkeld. Dit kader ondersteunt projectmanagers en architecten bij het maken van weloverwogen keuzes over de inzet van Mendix in vergelijking met conventionele ontwikkelmethoden. Het biedt beslissingspunten voor de initiële keuze tussen Mendix en traditionele ontwikkeling en stelt criteria vast om te bepalen welke projectonderdelen beter geschikt zijn voor high-code ontwikkeling. Daarnaast worden richtlijnen geformuleerd voor het effectief combineren van low-code en high-code in hybride projecten, zodat beide benaderingen optimaal kunnen worden ingezet. Verder bevat het kader aanbevelingen voor de vroege identificatie van potentiële beperkingen en risico’s, waardoor projectrisico’s proactief kunnen worden gemitigeerd. Tot slot worden strategieën ontwikkeld om op basis van eerdere ervaringen en literatuur weloverwogen keuzes te maken bij toekomstige projecten. Dit beslissingskader draagt bij aan een systematische en onderbouwde aanpak voor het selecteren van de meest geschikte ontwikkelmethode binnen uiteenlopende projectcontexten.



    %%=============================================================================
%% Methodologie
%%=============================================================================

\chapter{\IfLanguageName{dutch}{Huidige gang van zaken}{Results}}%
\label{ch:gangVanZaken}

%%  In dit hoofstuk geef je een korte toelichting over hoe je te werk bent
%% gegaan. Verdeel je onderzoek in grote fasen, en licht in elke fase toe wat
%% de doelstelling was, welke deliverables daar uit gekomen zijn, en welke
%% onderzoeksmethoden je daarbij toegepast hebt. Verantwoord waarom je
%% op deze manier te werk gegaan bent.
%% 
%% Voorbeelden van zulke fasen zijn: literatuurstudie, opstellen van een
%% requirements-analyse, opstellen long-list (bij vergelijkende studie),
%% selectie van geschikte tools (bij vergelijkende studie, "short-list"),
%% opzetten testopstelling/PoC, uitvoeren testen en verzamelen
%% van resultaten, analyse van resultaten, ...
%%
%% !!!!! LET OP !!!!!
%%
%% Het is uitdrukkelijk NIET de bedoeling dat je het grootste deel van de corpus
%% van je bachelorproef in dit hoofstuk verwerkt! Dit hoofdstuk is eerder een
%% kort overzicht van je plan van aanpak.
%%
%% Maak voor elke fase (behalve het literatuuronderzoek) een NIEUW HOOFDSTUK aan
%% en geef het een gepaste titel.

\section{Analyse van historische projectdata}
In de samenwerking tussen apvine en Federale Overheidsdienst Mobiliteit (FOD MOB), een klant met een intern Java-developmentteam, komt de keuze tussen high-code en low-code regelmatig terug als een belangrijk discussiepunt. Het ontbreken van een duidelijk en gestructureerd beslissingskader bemoeilijkt het maken van weloverwogen technologisch onderbouwde keuzes, wat vaak leidt tot suboptimale beslissingen. Dit maakt \gls{FOD MOB} tot een uitstekend casevoorbeeld om de gevolgen van besluiteloosheid en het ontbreken van heldere richtlijnen in het technologiekeuzeproces te analyseren. In dit onderzoek worden initiële projectschattingen vergeleken met de werkelijke uitkomsten, met als doel de impact op kosten en doorlooptijd nauwkeurig te kwantificeren. Daarnaast wordt de projectdocumentatie systematisch doorgenomen om concrete momenten te identificeren waarop de keuze tussen low-code en high-code tot frictie heeft geleid. Het onderzoek inventariseert ook in welke gevallen projectteams, bijvoorbeeld binnen \gls{FOD MOB}, gedwongen waren om tijdens het project over te schakelen naar alternatieve ontwikkelmethoden, als gevolg van onvoorziene beperkingen van het gekozen platform of ontwikkelparadigma. De financiële en tijdsimpact van deze late aanpassingen wordt vervolgens grondig geëvalueerd, om de werkelijke kosten van een gebrekkig technologisch beslissingskader inzichtelijk te maken.
\\
\\
Het ontbreken van een duidelijk beslissingskader heeft aanzienlijke gevolgen voor de technologische keuzes binnen de samenwerking tussen apvine en \gls{FOD MOB}. In plaats van op objectieve criteria en een gestructureerd evaluatieproces te vertrouwen, worden de keuzes voor high-code versus low-code vaak beïnvloed door informele voorkeuren en machtsverhoudingen binnen het team. Dit gebrek aan een formele en transparante methodologie leidt tot verschillende problemen die de effectiviteit van het project kunnen ondermijnen:
\subsubsection{Onzekerheid in de technologiekeuze}
Zonder een duidelijk beslissingskader is het voor betrokkenen moeilijk om met zekerheid te bepalen welke technologie het beste past bij de specifieke projectvereisten. Dit leidt vaak tot keuzes die niet goed afgestemd zijn op de werkelijke behoeften van het project, wat in de praktijk kan resulteren in inefficiëntie en suboptimale resultaten.
\subsubsection{Frictie binnen het projectteam}
Wanneer er geen duidelijke richtlijnen zijn voor het maken van technologische keuzes, ontstaat er vaak frictie binnen het projectteam. Teamleden hebben mogelijk verschillende opvattingen over de voor- en nadelen van low-code versus high-code, wat leidt tot onduidelijkheid en conflicten. Dit vergroot de kans op vertragingen, aangezien beslissingen niet snel genomen kunnen worden door gebrek aan consensus.
\subsubsection{Vertraagde besluitvorming}
Het proces van het maken van een keuze wordt uitgesteld of herhaald, omdat er geen objectief beslissingskader is om de discussie te structureren. Deze vertragingen kunnen het project belemmeren, vooral in de vroege fasen waarin cruciale technologische keuzes gemaakt moeten worden. Dit vertraagt de voortgang van het project en kan leiden tot hogere kosten en een langere doorlooptijd.
\subsubsection{Noodzaak tot herziening van keuzes}
Wanneer er geen formele afstemming en evaluatie plaatsvindt, worden technologische keuzes vaak pas later in het project beoordeeld. Dit kan betekenen dat gekozen technologieën niet voldoen aan de functionele eisen, schaalbaarheid of onderhoudsvereisten. Het projectteam moet dan dure en tijdrovende herzieningen doorvoeren, zoals het aanpassen van architectuur of het implementeren van alternatieve technologieën. Dit zorgt ervoor dat het project verder vertraagt en de kosten verhogen.
\\
\\
Kortom, het ontbreken van een helder beslissingskader maakt het lastig om weloverwogen keuzes te maken, verhoogt de kans op interne conflicten en vertragingen. Dit leidt uiteindelijk tot een verhoogd risico op het maken van suboptimale technologische keuzes die de voortgang en het succes van het project belemmeren.

\section{Interview met beslissingsmaker binnen \gls{FOD MOB}}

Binnen dit interview wordt een beslissingsmaker van \gls{FOD MOB} bevraagd over het keuzeproces omtrent technologische oplossingen in projecten waarbij zowel low-code als high-code mogelijkheden relevant zijn. In dit interview wordt één beslissingsmaker geïnterviewd, maar spreekt voor de hele instantie (\gls{FOD MOB}). \gls{FOD MOB} beschikt over een intern Java-developmentteam, wat impliceert dat er aanzienlijke technische expertise aanwezig is. Het is dan ook van belang te begrijpen hoe deze instantie omgaat met technologische voorstellen van externe partijen, zeker wanneer deze een voorkeur uitspreken voor low-code ontwikkeling.

Dit interview focust op de formele en informele besluitvormingsprocessen binnen \gls{FOD MOB}, met bijzondere aandacht voor de criteria die bepalend zijn voor de uiteindelijke keuze van technologie. Er wordt onderzocht in welke mate beslissingen worden genomen op basis van objectieve evaluaties zoals technische haalbaarheid, onderhoudsimpact en langetermijnvisie. Of hoe hard deze beslissingen beïnvloed worden door andere factoren zoals strategische belangen, tijdsdruk of machtsverhoudingen binnen projectstructuren.

Daarnaast wordt verkend hoe de instantie omgaat met situaties waarin de aanvankelijk gekozen technologie gaandeweg minder geschikt blijkt te zijn. Daarbij is het van belang te achterhalen of er binnen de organisatie ruimte bestaat om bij te sturen, en welke lessen eerder genomen beslissingen hebben opgeleverd voor toekomstige projecten.

\subsection{Interviewvragen voor beslissingsmaker binnen \gls{FOD MOB}}
\begin{itemize}
    \item Kunt u beschrijven hoe binnen \gls{FOD MOB} het keuzeproces voor een technologie doorgaans verloopt bij de start van een project?
    \item Welke criteria zijn voor u doorslaggevend bij de keuze tussen low-code en high-code oplossingen?
    \item Worden deze criteria expliciet vastgelegd of zijn ze vooral impliciet aanwezig in het beslissingsproces?
    \item In hoeverre worden technologische beslissingen ondersteund door objectieve evaluaties, zoals technische haalbaarheidsstudies?
    \item Kunt u een voorbeeld geven van een project waarin een gekozen technologie (bijv. low-code) achteraf niet optimaal bleek? Hoe werd hiermee omgegaan?
    \item Hoe beoordeelt u de mate van flexibiliteit binnen \gls{FOD MOB} om in de loop van een project de technologische koers bij te stellen indien nodig?
    \item Zou een formeel en gestructureerd beslissingskader volgens u een meerwaarde kunnen bieden bij het maken van technologische keuzes binnen toekomstige projecten? Waarom wel of niet?
\end{itemize}

\subsection{Analyse van de interview met beslissingsmaker binnen \gls{FOD MOB}}

Om inzicht te krijgen in het besluitvormingsproces rond technologische keuzes binnen \gls{FOD MOB}, werd een interview afgenomen met een beslissingsmaker die betrokken is bij projecten waarin zowel low-code als high-code opties overwogen worden. De antwoorden werden thematisch geanalyseerd aan de hand van terugkerende patronen.

\subsubsection{Technologiekeuze bij projectstart}

Uit het interview blijkt dat het keuzeproces voor een technologie doorgaans informeel verloopt. Hoewel er bij de start van elk project een technische analyse wordt uitgevoerd, bestaat er geen vastgelegd beslissingskader dat richting geeft aan de keuze tussen low-code en high-code. Zo gaf de geïnterviewde aan: \textit{“We evalueren vooral op basis van onze interne expertise en de complexiteit die we verwachten. Daarnaast spelen ook de scope van het project en de beschikbaarheid van onze teams een grote rol. Het is echter geen vastgelegde procedure.”} (L. Debusscher, persoonlijke communicatie, 28 mei 2025).

In de praktijk gebeurt de besluitvorming meestal via overleg tussen de development managers van zowel Mendix als Java, aangevuld met enkele softwarearchitecten. Deze groep beslist samen op basis van eerdere ervaringen, inschattingen van technische haalbaarheid en de context van het project, zonder dat dit formeel is vastgelegd of systematisch wordt geëvalueerd. Zoals de geïnterviewde toelichtte: \textit{“De keuze hangt vaak sterk af van het departement. Teams die al vertrouwd zijn met low-code en Mendix, kiezen daar meestal ook voor. Andere teams, die traditioneel met Java werken, blijven vaak bij die technologie.”} (L. Debusscher, persoonlijke communicatie, 28 mei 2025).

\subsubsection{Besluitvormingscriteria}

De geïnterviewde benadrukt dat er geen vaste besluitvormingscriteria worden gehanteerd bij de keuze voor een technologie. Hoewel factoren zoals technische haalbaarheid, integratiemogelijkheden en langetermijnonderhoud een rol spelen, berust de beslissing vooral op intuïtie en ervaring. Ook minder tastbare aspecten zoals projectdruk en voorkeuren van externe partners beïnvloeden de keuze. Zoals de geïnterviewde opmerkte: \textit{“Er is wel een procedure die we volgen, maar er zijn eigenlijk geen duidelijke criteria waarop we onze keuze baseren.”} (L. Debusscher, persoonlijke communicatie, 28 mei 2025).

\subsubsection{Bijsturen tijdens het project}

In de praktijk wordt bijsturen tijdens een project zelden toegepast. Overstappen van de gekozen technologie, bijvoorbeeld van low-code naar een custom Java-oplossing, komt nauwelijks voor vanwege de hoge kosten en het aanzienlijke tijdsverlies dat hiermee gepaard gaat. Hierdoor is het vrijwel onmogelijk om halverwege het project nog van koers te veranderen zonder grote impact op planning en budget. Dit maakt de initiële besluitvorming over de te gebruiken technologie des te belangrijker, omdat de keuze in een vroeg stadium goed doordacht en zorgvuldig moet worden gemaakt om later kostbare aanpassingen te voorkomen.


\subsubsection{Nood aan een beslissingskader}

De geïnterviewde erkent het belang en het potentieel van een gestructureerd beslissingskader binnen het keuzeproces. Hoewel het resultaat van zo’n kader niet als een definitief besluit moet worden gezien, maar vooral dient als motivatie voor een gemaakte keuze, is er tegelijkertijd ook grote interesse in de uitkomsten ervan. Zoals de geïnterviewde aangeeft: \textit{“Het resultaat van het beslissingskader mag niet dienen als een effectief besluit, maar meer als motivatie voor een keuze.”} (L. Debusscher, persoonlijke communicatie, 28 mei 2025). Daarnaast benadrukt hij dat het kader voldoende flexibel moet zijn om rekening te houden met de unieke context en specifieke omstandigheden van elk project, zodat het bruikbaar blijft in diverse situaties.





\section{Documentatieonderzoek}

Uit de analyse van de interne documentatie blijkt dat de keuze tussen low-code en high-code ontwikkeling momenteel niet is verankerd in duidelijke en formeel vastgelegde richtlijnen. In plaats daarvan lijkt deze beslissing grotendeels te worden genomen op basis van intuïtie, eerdere ervaringen en de directe beschikbaarheid van zowel interne als externe ontwikkelteams. Hoewel er in documenten zoals pre-sales offertes, kick-off verslagen en architectuurbesluiten wel verwijzingen zijn naar technologiekeuzes, ontbreekt een gestructureerd afwegingskader dat systematisch wordt toegepast. Hierdoor ontstaat het risico dat keuzes niet altijd consistent of onderbouwd zijn, en dat belangrijke contextuele factoren zoals schaalbaarheid, onderhoudbaarheid of klantbehoeften, onvoldoende worden meegewogen in het besluitvormingsproces.
\\
Een bijkomende observatie is dat de mate van personalisatie van de applicatie eveneens een doorslaggevende rol speelt in de keuze voor de ontwikkelmethode. Wanneer een diep gepersonaliseerde oplossing vereist is, wordt vaker gekozen voor het interne Java-team, vanwege hun mogelijkheid om maatwerk te leveren. Daarentegen wordt bij projecten met een strakke deadline of een minder complexe functionaliteitsbehoefte vaak gekozen voor een low-code platform, omdat hiermee sneller resultaat kan worden geboekt. Deze pragmatische aanpak is begrijpelijk, maar onderstreept tegelijkertijd het ontbreken van expliciete richtlijnen die dergelijke afwegingen structureren.








    %%=============================================================================
%% Methodologie
%%=============================================================================

\chapter{\IfLanguageName{dutch}{Hoe kan het beter?}{Results}}%
\label{ch:beter}



\section{Literatuuroverzicht}
TODO
\\
\\
Een diepgaande analyse van bestaande documentatie over Mendix en andere low-code platforms vormt een essentieel onderdeel voor het beantwoorden van deelvraag 5 en 7. Deze analyse omvat diverse bronnen, waaronder officiële Mendix productgidsen die gedetailleerde technische specificaties en functionaliteiten beschrijven, casestudies van derden en rapporten uit de industrie die praktijkvoorbeelden en onafhankelijke evaluaties bieden, online forums en ontwikkelaarsgemeenschappen waar praktijkervaringen en knelpunten worden gedeeld, en academische publicaties over low-code ontwikkeling die theoretische onderbouwing en wetenschappelijke inzichten verschaffen over de bredere context van deze technologie.
\section{Praktisch onderzoek}
TODO
\\
\\
\subsection{Schaalbaarheid en prestaties}
TODO
\\
\\
Hoewel specifieke benchmarks voor reactietijd en belastingstests niet publiekelijk beschikbaar zijn, is bekend dat Mendix is ontworpen voor cloud-native en schaalbare toepassingen. Het platform ondersteunt containerisatie en multi-cloud-deployments, wat bijdraagt aan hoge beschikbaarheid en schaalbaarheid. Mendix garandeert een uptime van 99,95\% voor premium versies, wat duidt op een robuuste infrastructuur die geschikt is voor toepassingen met hoge beschikbaarheid.

\subsection{Integratiemogelijkheden}
Mendix is ontworpen om gemakkelijk te kunnen integreren met zowel moderne als legacy systemen. De belangrijkste integratietechnieken en mogelijkheden zijn:
\subsubsection{REST API's}
Mendix biedt volledige ondersteuning voor het bouwen en gebruiken van RESTful API's. Je kunt makkelijk data van externe systemen ophalen of je eigen services beschikbaar maken voor anderen.
Bijvoorbeeld: integreren met een CRM zoals Salesforce of een ERP zoals SAP.
\subsubsection{SOAP Webservices}
Ook oudere webservices (SOAP) worden ondersteund, wat belangrijk is voor legacy systemen.
Bijvoorbeeld: communicatie met oudere financiële software of verzekeringssystemen.

\subsubsection{GraphQL ondersteuning}
Hoewel GraphQL niet natively wordt aangeboden zoals REST/SOAP, kan Mendix met plugins en uitbreidingen GraphQL endpoints benaderen.

\subsubsection{OData Services}
Voor standaardisatie en eenvoudige integratie met analytische tools (bijvoorbeeld Power BI) ondersteunt Mendix OData publicatie en consumptie.
\\
\\
Mendix biedt uitgebreide integratiemogelijkheden met externe systemen, API’s en databases. Het platform ondersteunt integratie via REST, SOAP, OData, GraphQL en JDBC. Daarnaast kunnen ontwikkelaars gebruikmaken van Mendix Connect en de Mendix Marketplace om connectors te vinden of te bouwen voor systemen zoals SAP, Salesforce en Kafka. Voor database-integratie biedt Mendix ondersteuning voor directe JDBC-verbindingen met databases zoals Microsoft SQL, MySQL, Oracle, PostgreSQL en Snowflake.

\subsection{Aanpasbaarheid en uitbreidbaarheid}
Het Mendix-platform is zeer aanpasbaar en uitbreidbaar. Ontwikkelaars kunnen aangepaste functionaliteiten implementeren via Java-acties, pluggable widgets (op basis van React of JavaScript) en de Mendix Model SDK. Daarnaast kunnen ontwikkelaars het platform uitbreiden met behulp van de Extensibility API, waarmee nieuwe functionaliteiten aan Studio Pro kunnen worden toegevoegd.

\subsection{Ontwikkelingssnelheid}
Mendix staat bekend om zijn snelle ontwikkelingsmogelijkheden. Het platform maakt gebruik van een visuele ontwikkelomgeving, herbruikbare componenten en een geïntegreerde CI/CD-pijplijn, wat bijdraagt aan een verhoogde ontwikkelsnelheid. Volgens \textcite{MxTechies} kunnen applicaties tot 10 keer sneller worden gebouwd met 70\% minder middelen. 

\subsection{Beveiliging en compliance}
Mendix hecht veel waarde aan beveiliging en compliance. Het platform biedt uitgebreide beveiligingsmaatregelen, waaronder toegangsbeheer, gegevensversleuteling en ondersteuning voor beveiligingsstandaarden. Daarnaast ondersteunt Mendix de opslag van applicatiegegevens in SQL-databases naar keuze, wat organisaties flexibiliteit biedt in gegevensbeheer en compliance. 
\\
\\
\subsection{Conclusie}
De uitgevoerde tests tonen aan dat Mendix een krachtig en flexibel low-code platform is dat geschikt is voor het bouwen van schaalbare, geïntegreerde en veilige applicaties. De uitgebreide integratiemogelijkheden, aanpasbaarheid en snelle ontwikkelcyclus maken het platform bijzonder geschikt voor diverse bedrijfsbehoeften.


\section{Reflectie op eigen ervaringen}
Op basis van mijn ervaring kan ik bevestigen dat low-code, zoals Mendix, bijzonder krachtig is voor het snel opzetten van generieke applicaties met standaardfunctionaliteiten. Het stelt je in staat om in korte tijd werkende oplossingen te bouwen, wat vooral in iteratieve of proof-of-concept contexten een grote meerwaarde biedt. Tegelijk merk ik dat zodra een project meer ‘custom’ noden heeft – zoals complexe bedrijfslogica of verfijnde integraties – de grenzen van het platform sneller voelbaar worden. In die gevallen is het een duidelijke troef om een high-code achtergrond te hebben: je begrijpt beter wat er onder de motorkap gebeurt, kunt gerichter zoeken naar workarounds en neemt bewuster beslissingen over de architectuur van je oplossing. Bovendien zie ik dat een traditionele programmeerachtergrond ook de leesbaarheid en structuur van je low-code logica ten goede komt. Je denkt in patronen, zorgt voor herbruikbaarheid en hanteert best practices die niet vanzelfsprekend zijn in een puur visuele ontwikkelomgeving.

\subsection{Reflectie op ervaringen van experts}
Er zijn ook enkele ontwikkelaars met een klassieke high-code achtergrond die inmiddels volledig actief zijn binnen low-code projecten, met name op het Mendix-platform. Ik sprak met hen over hun ervaringen en vatte hun bevindingen samen. 
Ze benadrukken dat low-code bijzonder krachtig is voor het snel ontwikkelen van generieke applicaties met standaardfunctionaliteiten. Dit maakt het ideaal voor situaties waarin snelheid en iteratieve ontwikkeling belangrijk zijn. Daarnaast wordt low-code vaak gepositioneerd als een brug tussen IT en business, doordat ook gebruikers zonder programmeerervaring relatief snel aan de slag kunnen. In de praktijk leidde dit er echter soms toe dat businessgebruikers eigen applicaties opstartten die later door ervaren ontwikkelaars moesten worden overgenomen. Deze overdracht bleek niet altijd evident: de onderliggende logica bleek vaak moeilijk leesbaar en voldeed zelden aan gangbare ontwikkelstandaarden of best practices, wat extra werk met zich meebracht om de applicatie te stabiliseren en verder te ontwikkelen.

Wat betreft de ontwikkelervaring binnen Mendix, werd er gemengd gereageerd op de version control-functionaliteit. Hoewel het systeem in principe krachtig is en goed integreert met teamwerk, kunnen foutmeldingen en merge-conflicten soms moeilijk te doorgronden zijn. Wanneer alles echter correct functioneert, biedt het versiebeheer een betrouwbare en efficiënte manier van samenwerken. De integratie van agile werkmethodieken binnen het Mendix-platform werd unaniem positief beoordeeld: user stories, sprints en voortgang kunnen rechtstreeks via de projectpagina opgevolgd en beheerd worden, wat de samenwerking tussen ontwikkelaars en stakeholders vergemakkelijkt.

Ook het gebruik van herbruikbare modules uit de Mendix Marketplace werd als een groot voordeel genoemd. Het toevoegen van bestaande componenten versnelt de ontwikkeling aanzienlijk en voorkomt dat het wiel telkens opnieuw moet worden uitgevonden. Tegelijk wordt opgemerkt dat een groot aantal van deze modules weinig tot geen documentatie bevat, waardoor het tijd kost om hun werking te doorgronden of aan te passen aan specifieke projectbehoeften. 

Over het geheel genomen beschouwen deze ontwikkelaars Mendix als een toegankelijke en efficiënte ontwikkelomgeving, die eenvoudig aanvoelt in de basis, maar bij complexere noden ook de nodige technische diepgang vereist. Hun ervaring met high-code vormt daarbij een duidelijke meerwaarde: het helpt hen om concepten sneller te begrijpen, kwalitatieve oplossingen te bouwen en de leesbaarheid en onderhoudbaarheid van hun low-code toepassingen te verbeteren.

\section{Evaluatie van Mendix-uitbreidingen}
Op basis van gesprekken met ervaren ontwikkelaars die dagelijks met Mendix werken, blijkt dat het gebruik van standaard Marketplace-modules doorgaans als efficiënt en tijdbesparend wordt ervaren, vooral bij generieke functionaliteiten. Deze modules kunnen snel geïntegreerd worden, wat de implementatietijd aanzienlijk verkort. Toch werd ook opgemerkt dat veel van deze modules onvoldoende of zelfs geheel geen documentatie bevatten. Dit gebrek aan transparantie leidt tot vertragingen tijdens implementatie en beperkt de flexibiliteit wanneer aanpassingen nodig zijn. In zulke gevallen biedt custom ontwikkeling vaak meer controle en beter afgestemde oplossingen, hoewel dit uiteraard gepaard gaat met een langere ontwikkeltijd en hogere initiële kosten.

Java-uitbreidingen binnen Mendix worden door ontwikkelaars met een high-code achtergrond beschouwd als waardevolle tools om de beperkingen van het platform te omzeilen, bijvoorbeeld bij complexe logica of integraties met externe systemen. Deze uitbreidingen verhogen echter ook de technische complexiteit van het project en kunnen de onderhoudbaarheid op lange termijn negatief beïnvloeden, zeker wanneer ze niet goed gedocumenteerd zijn of slechts door een beperkte groep binnen het team begrepen worden.

Wat betreft de kosten-batenverhouding, geven ontwikkelaars aan dat low-code in combinatie met herbruikbare modules initieel voordeliger is in zowel ontwikkeling als beheer. Naarmate de applicatie complexer wordt en meer ‘maatwerk’ vereist, verschuift dit evenwicht echter, en kunnen custom uitbreidingen of Java-integraties duurder uitvallen, zowel in termen van ontwikkeluren als bij latere aanpassingen of onderhoud.

Tot slot werd ook het integreren van externe systemen als een uitdaging benoemd. Hoewel Mendix hier voldoende ondersteuning voor biedt, vergt het combineren van low-code met externe services een goed begrip van beide kanten. Een technische achtergrond blijkt in die context bijzonder waardevol, zowel voor het begrijpen van de externe API’s als voor het bouwen van robuuste, onderhoudbare koppelingen binnen het platform. Al met al tonen deze bevindingen aan dat een doordachte strategie, met oog voor schaalbaarheid en onderhoud, essentieel is bij het uitbreiden van Mendix-toepassingen.

\section{Ontwikkeling beslissingskader}
TODO
\\
\\
Voor het beantwoorden van deelvraag 8, die zich richt op het destilleren van best practices en richtlijnen voor de keuze tussen Mendix en traditionele ontwikkeling, wordt een gestructureerd beslissingskader ontwikkeld. Dit kader ondersteunt projectmanagers en architecten bij het maken van weloverwogen keuzes over de inzet van Mendix in vergelijking met conventionele ontwikkelmethoden. Het biedt beslissingspunten voor de initiële keuze tussen Mendix en traditionele ontwikkeling en stelt criteria vast om te bepalen welke projectonderdelen beter geschikt zijn voor high-code ontwikkeling. Daarnaast worden richtlijnen geformuleerd voor het effectief combineren van low-code en high-code in hybride projecten, zodat beide benaderingen optimaal kunnen worden ingezet. Verder bevat het kader aanbevelingen voor de vroege identificatie van potentiële beperkingen en risico’s, waardoor projectrisico’s proactief kunnen worden gemitigeerd. Tot slot worden strategieën ontwikkeld om op basis van eerdere ervaringen en literatuur weloverwogen keuzes te maken bij toekomstige projecten. Dit beslissingskader draagt bij aan een systematische en onderbouwde aanpak voor het selecteren van de meest geschikte ontwikkelmethode binnen uiteenlopende projectcontexten.



    
    % Voeg hier je eigen hoofdstukken toe die de ``corpus'' van je bachelorproef
    % vormen. De structuur en titels hangen af van je eigen onderzoek. Je kan bv.
    % elke fase in je onderzoek in een apart hoofdstuk bespreken.
    
    %\input{...}
    %\input{...}
    %...
    
    %%=============================================================================
%% Conclusie
%%=============================================================================

\chapter{Conclusie}%
\label{ch:conclusie}

% Trek een duidelijke conclusie, in de vorm van een antwoord op de
% onderzoeksvra(a)g(en). Wat was jouw bijdrage aan het onderzoeksdomein en
% hoe biedt dit meerwaarde aan het vakgebied/doelgroep? 
% Reflecteer kritisch over het resultaat. In Engelse teksten wordt deze sectie
% ``Discussion'' genoemd. Had je deze uitkomst verwacht? Zijn er zaken die nog
% niet duidelijk zijn?
% Heeft het onderzoek geleid tot nieuwe vragen die uitnodigen tot verder 
%onderzoek?

Deze bachelorproef had als centrale vraag: “Wat zijn de grenzen van de low-code-tool Mendix bij het ontwikkelen van zakelijke applicaties, en wanneer is high-code-ontwikkeling meer geschikt?” Aan de hand van deze hoofdvraag werden acht deelvragen onderzocht, waarvan hieronder de belangrijkste bevindingen worden samengevat.

Ten eerste leidt het ontbreken van een gestructureerd beslissingskader bij verschillen bedrijven tot een informeel keuzeproces dat sterk afhankelijk is van de ervaring en voorkeuren van individuele teams. Dit veroorzaakt onzekerheden en inconsistenties in de technologische keuzes binnen projecten, wat de effectiviteit en efficiëntie negatief beïnvloedt. Door het ontbreken van richtlijnen ontstaan er problemen zoals inefficiënte besluitvorming en beperkte mogelijkheden om tijdens projecten bij te sturen.

De huidige criteria die bijvoorbeeld \gls{FOD MOB} gebruikt bij het kiezen tussen low-code en high-code zijn vooral gebaseerd op interne expertise, de verwachte complexiteit van het project, de scope, en de beschikbaarheid van teams. Projectmanagers en architecten ervaren hierbij knelpunten doordat deze criteria weinig systematisch en expliciet zijn vastgelegd, wat het risico op suboptimale keuzes verhoogt.

Technisch gezien blijkt Mendix beperkingen te hebben bij het omgaan met complexe bedrijfslogica, intensieve real-time verwerking en integraties met legacy-systemen. In deze situaties kan een hybride oplossing of volledige high-code ontwikkeling noodzakelijk zijn. De triggers voor een overstap zijn vaak gerelateerd aan de complexiteit van de functionele eisen en technische haalbaarheid.

Daarnaast spelen projectomvang, tijdslijnen en klantvereisten een belangrijke rol bij de keuze voor low-code of high-code. Strakke deadlines en omvangrijke projecten vragen soms om de flexibiliteit van high-code oplossingen. Uit eerdere projecten bij apvine blijkt dat het zorgvuldig afwegen van deze factoren essentieel is om vertragingen en extra kosten te voorkomen.

Op basis van deze inzichten kan worden geconcludeerd dat een gestructureerd maar flexibel beslissingskader waardevol is voor zowel apvine als hun klanten. Dit kader moet niet als starre richtlijn fungeren, maar als een hulpmiddel dat keuzes motiveert en ondersteunt, rekening houdend met zowel technische als organisatorische contexten.

\bigskip

Hoewel het onderzoek duidelijk maakt dat een beslissingskader broodnodig is, roept het ook vragen op over de praktische implementatie ervan binnen bedrijven. De huidige afhankelijkheid van ervaring en ‘buikgevoel’ suggereert dat een cultuurverandering noodzakelijk is om een dergelijk kader breed te accepteren en effectief te gebruiken. Tevens is het onzeker of een uniform kader kan inspelen op de diversiteit van projecten en teams zonder aan flexibiliteit in te boeten. Het fragmenteren van technologievoorkeuren tussen teams kan een barrière vormen bij de invoering van gemeenschappelijke richtlijnen.

Verder laat het onderzoek zien dat de dynamiek van projecten, klantwensen en technologische ontwikkelingen continue aanpassingen van het kader vereisen. Dit betekent dat een beslissingskader niet een eenmalig product is, maar een levend document dat regelmatig geëvalueerd en bijgesteld moet worden. Ook is het van belang om verandermanagement en communicatie mee te nemen in het invoeringsproces, om draagvlak te creëren en weerstand te minimaliseren.

Kortom, de studie bevestigt het belang van een doordacht keuzeproces en het potentieel van een ondersteunend kader, maar benadrukt ook dat succes afhankelijk is van een goede afstemming op organisatiecultuur, flexibiliteit en continue verbetering.


    
    %---------- Bijlagen -----------------------------------------------------------
    
    \appendix
    
    \chapter{Onderzoeksvoorstel}
    
    Het onderwerp van deze bachelorproef is gebaseerd op een onderzoeksvoorstel dat vooraf werd beoordeeld door de promotor. Dat voorstel is opgenomen in deze bijlage.
    
    
    \section*{Samenvatting}
    
    % Kopieer en plak hier de samenvatting (abstract) van je onderzoeksvoorstel.
    In het digitale landschap van vandaag zoeken bedrijven steeds naar manieren om zakelijke applicaties efficiënter te ontwikkelen. Low-code ontwikkelplatformen, zoals Mendix, zijn de laatste jaren sterk gegroeid als een vervanging voor traditionele high-code ontwikkeling. Deze platformen beloven snellere ontwikkelingstijden en toegankelijkheid door visuele interfaces en herbruikbare componenten. Er is echter nog onduidelijkheid over wat de grenzen van de low-code-tool Mendix zijn bij het ontwikkelen van zakelijke applicaties, en wanneer high-code-ontwikkeling meer geschikt is.
    \\
    Dit onderzoek duikt diep in de technische mogelijkheden en beperkingen van Mendix om deze centrale vraag te beantwoorden. Door de complexiteit van deze vraag te ontrafelen, zal de studie een genuanceerd beeld schetsen van de huidige stand van low-code technologie.
    \\
    Een gemengde onderzoeksmethodologie zal worden gehanteerd, bestaande uit een diepgaande literatuurstudie, praktische experimenten en interviews met ervaren ontwikkelaars. Deze veelzijdige onderzoeksaanpak stelt ons in staat om de technische grenzen van Mendix grondig te analyseren, de prestaties in verschillende scenario's te evalueren en concrete richtlijnen te formuleren voor organisaties die overwegen low-code te implementeren.
    \\
    Het onderzoek zal een gedetailleerde analyse geven van de technische beperkingen van Mendix en use cases identificeren waarin de ontwikkeling van high-code meer geschikt is. Het zal ook een beslissingskader bieden, zodat bedrijven en andere instellingen kunnen kiezen tussen low-code en high-code methoden. Uiteindelijk zal het richtlijnen bevatten voor het optimaal combineren van de twee methoden om de beste resultaten te behalen.
    \\
    Gezien de groeiende adoptie van low-code platforms is er dringend behoefte aan objectief onderzoek naar hun mogelijkheden en beperkingen. Deze bachelorproef helpt om deze behoefte aan kennis te vervullen en biedt handige adviezen voor organisaties die overwegen om low-code of traditionele ontwikkelmethoden toe te passen.
    
    % Verwijzing naar het bestand met de inhoud van het onderzoeksvoorstel
    %---------- Inleiding ---------------------------------------------------------

\section{Inleiding}%
\label{sec:inleiding}

Apvine, een toonaangevend IT-consultancybedrijf, richt zich op het creëren van applicaties met low-code platforms, voornamelijk Mendix. Deze aanpak is zeer effectief gebleken voor de meeste van hun projecten en maakt snelle ontwikkeling en implementatie mogelijk. Toch zijn er situaties waarin projecten uitdagingen bieden die de grenzen van low-code platforms opzoeken. Denk hierbij aan ingewikkelde bedrijfslogica, intensieve real-time verwerking of complexe integraties met legacy-systemen, die de mogelijkheden van een puur low-code methode kunnen overstijgen.
\\
\\
In dergelijke situaties kan het noodzakelijk zijn om over te stappen op een hybride of high-code methode. Het is echter niet eenvoudig om te beslissen wanneer deze overgang moet gebeuren. Zonder duidelijke richtlijnen loopt Apvine het risico op vertragingen, hogere uitgaven en ontevreden klanten door late of onverwachte aanpassingen in de ontwikkelingsstrategie.
Om deze valkuilen te vermijden, richt dit onderzoek zich op de centrale vraag:
“Wat zijn de grenzen van de low-code-tool Mendix bij het ontwikkelen van zakelijke applicaties, en wanneer is high-code-ontwikkeling meer geschikt?”
\\
\\
Dit onderzoek heeft als doel een raamwerk voor besluitvorming te ontwikkelen dat projectmanagers en architecten van Apvine ondersteunt bij het beoordelen of ze moeten blijven met low-code of overstappen naar high-code voor een specifiek project. Het raamwerk zal worden gebaseerd op:
\begin{itemize}
    \item Systematische analyse van eerdere projecten om gemeenschappelijke succesfactoren en uitdagingen te identificeren.
    \item Richtlijnen voor het implementeren van hybride methoden om zowel flexibiliteit als complexiteit te combineren.
    \item Een hulpmiddel dat beslissingen ondersteunt tijdens de fases van pre-sales en planning.
\end{itemize}
Om de centrale vraag te beantwoorden, worden de volgende deelvragen onderzocht:

\textbf{Probleemdomein:}
\begin{enumerate}
    \item Wat zijn de gevolgen van het ontbreken van een beslissingskader voor de keuze tussen low-code en high-code bij Apvine?
    \item Welke problemen ontstaan er in projecten door het gebrek aan richtlijnen voor de keuze tussen ontwikkelmethoden?
    \item Wat zijn de huidige criteria die Apvine gebruikt bij het kiezen tussen low-code en high-code ontwikkeling?
    \item Welke knelpunten ervaren projectmanagers en architecten bij het maken van de keuze tussen low-code en high-code?
\end{enumerate}


\textbf{Oplossingsdomein:}
\begin{enumerate}
    \setcounter{enumi}{4}
    \item  Wat zijn de technische beperkingen van Mendix bij het omgaan met complexe bedrijfslogica, intensieve real-time verwerking en integraties met legacy-systemen?
    \item In welke specifieke scenario's binnen Mendix-projecten kan een hybride of high-code oplossing nodig zijn, en wat zijn de triggers voor het maken van deze overstap?
    \item  Hoe kunnen projectomvang, tijdslijnen en klantvereisten de beslissing beïnvloeden om wel of niet met Mendix (low-code) te werken, of over te schakelen naar high-code? 
    \item Welke lessen kunnen worden getrokken uit de ervaring van eerdere projecten bij Apvine waarin Mendix werd ingezet, en hoe kunnen deze inzichten helpen bij het bepalen wanneer een hybride of high-code oplossing nodig is?
\end{enumerate}

Het beslissingskader dat uit dit onderzoek voortkomt, biedt een gestructureerde aanpak om organisaties te helpen bij het maken van een weloverwogen keuze tussen low-code en high-code ontwikkelmethoden. Met dit kader zijn bedrijven zoals Apvine in staat om niet alleen effectiever te plannen, maar ook de risico's van late veranderingen in projecten te verlagen, de ontwikkelingskosten effectiever te beheren en hoogwaardige oplossingen te bieden die voldoen aan de unieke wensen van hun klanten.
\\
\\
\\

%---------- Stand van zaken ---------------------------------------------------

\section{Literatuurstudie}%
\label{sec:literatuurstudie}
\subsection{Low-code en Mendix}
Het landschap van digitale klantervaringen evolueert snel, aangewakkerd door de groeiende verwachtingen van klanten en de behoefte voor bedrijven om steeds consistente en hoogwaardige ervaringen te bieden. Zoals beschreven in de tekst “Deliver Standout Digital Customer Experiences with Low-Code” door het bedrijf \textcite{Mendix}, worden bedrijven geconfronteerd met aanzienlijke uitdagingen om aan deze eisen te voldoen vanwege de complexiteit van traditionele ontwikkelpraktijken en beperkte middelen.
\\
Ongelijksoortige processen en afzonderlijke oplossingen leiden vaak tot onsamenhangende ervaringen die niet aansluiten bij de wensen van de klant \autocite{Mendix}. Ondernemingen moeten worstelen met het beheer van meerdere complexe technologieën, afzonderlijke applicaties en een vertraagde time-to-market. Dit alles hindert hen erin om zich snel aan te passen en de ervaringen te bieden die klanten verwachten .
\\
Low-code ontwikkelplatforms zijn aangetoond als een effectieve oplossing voor deze problemen, waardoor bedrijven de controle kunnen nemen over de klantervaring. Low-code platformen maken zowel technische als niet-technische teams mogelijk om samen te werken aan de snelle ontwikkeling van multi-ervaringstoepassingen door snelle ontwikkelingskansen en een klantgerichte benadering te bieden.
\\
Met name het low-code Mendix-platform is ontworpen om bedrijven te ondersteunen bij digitalisering en om consistente, toekomstbestendige contactpunten voor klanten te creëren voor diverse kanalen en apparaten. Dankzij functies zoals automatisering, AI-ondersteuning en cloud-native schaalbaarheid stelt Mendix bedrijven in staat om snel applicaties te bouwen en te implementeren, terwijl ze tegelijkertijd flexibel en consistent blijven. ~\autocite{Mendix}

\subsection{Low-code use cases}
Low-code ontwikkelplatforms maken het voor organisaties mogelijk om een breed scala aan bedrijfsapplicaties te creëren, zodat ze hun bedrijfsprocessen kunnen faciliteren en de gebruikerservaringen kunnen optimaliseren.
Zo zijn er veelvoorkomende use cases:
\begin{itemize}
    \item Legacy modernisering 
    \\
    Met low-code hebben bedrijven de mogelijkheid om legacy-systemen te combineren met nieuwere technologieën, bestaande kansen te vergroten of verouderde systemen te vervangen om in te spelen op veranderende bedrijfsbehoeften. Banco de Occidente maakte gebruik van een low-code platform om hun verouderde systemen te integreren en hun processen te verbeteren, wat leidde tot een betere ervaring voor zowel klanten als medewerkers \autocite{Bunce2024}.
    \item Portalen
    \\
    Low-code stelt gebruikers in staat om op maat gemaakte, webgebaseerde portals te creëren die hen van belangrijke informatie en acties voorzien, wat de efficiëntie en gebruikerservaring ten goede komt. DHL Group heeft een platform ontwikkeld voor het beheer van leveranciersstamgegevens, gebaseerd op een low-code platform, waarmee het werk wordt geautomatiseerd. \autocite{Bunce2024} .
    \item Mobiele apps
    \\
    Low-code applicaties functioneren op diverse apparaten zonder dat er hercodering vereist is, waardoor gebruikers toegang krijgen tot het apparaat dat ze verkiezen. Super Bock Group maakte gebruik van low-code voor het ontwikkelen van een mobiel goedkeuringsproces voor aankoopaanvragen, wat resulteerde in kortere responstijden \autocite{Bunce2024a}.
    \item Integratie
    \\
    Low-code platforms maken het mogelijk om verschillende systemen en gegevensbronnen te integreren, zodat gebruikers in één interface toegang krijgen tot informatie. Zo maakte bijvoorbeeld Unilever gebruik van een low-code platform om hun \\SAP-organisaties en prijsstrategieën voor \\klanten te integreren, wat resulteerde in een hogere efficiëntie en precisie \autocite{Bunce2024}.
\end{itemize}
Dit zijn slechts een paar voorbeelden van de veelzijdigheid en brede toepassing van low-code ontwikkelplatformen in diverse industrieën en gebruikssituaties. Bij een volledige literatuurstudie kan/zal dit verder uitgewerkt worden.

\subsection{Beperkingen van low-code}
Hoewel low-code ontwikkelplatformen veel voordelen opleveren bij het versnellen van de applicatielevering, hebben ze ook enkele opvallende nadelen waar organisaties rekening mee moeten houden. Het artikel “9 Low Code Limitations in 2024 to Know About” \autocite{Malak2024} beschrijft verschillende belangrijke beperkingen die verband houden met low-code methoden. Deze beperkingen omvatten:

\begin{itemize}
    \item Een tekort aan controle over de automatisch genereerde code kan een obstakel vormen voor ingewikkelde applicaties die codeoptimalisatie vereisen.
    \item Potentiële vendor lock-in door eigen frameworks of programmeertalen maakt het lastig om van platform te veranderen of met andere systemen te integreren.
    \item Er zijn beperkte mogelijkheden voor aanpassing, omdat low-code platforms beter geschikt zijn voor algemene functies dan voor ingewikkelde, specifieke vereisten binnen een bepaalde sector.
    \item Beveiligingsproblemen ontstaan doordat low-code ontwikkeling sommige beveiligingsaspecten kan over het hoofd zien, vooral in streng gereguleerde sectoren.
    \item Complexiteit van integratie komt voort uit het feit dat low-code platforms mogelijk niet perfect kunnen integreren met de huidige infrastructuur van een organisatie en externe diensten.
    \item Beperkingen in de schaalbaarheid, omdat low-code applicaties problemen kunnen ondervinden bij het aanpassen aan de groeiende vraag van gebruikers en het volume aan gegevens.
\end{itemize}
De geschiktheid van low-code voor een specifiek project zal uiteindelijk afhankelijk zijn van elementen zoals de ingewikkeldheid van de applicatie, de technische expertise van de organisatie, de behoefte aan beveiliging en naleving, evenals de langetermijnschaalbaarheidseisen. Door de in dit artikel genoemde beperkingen grondig te beoordelen, kunnen bedrijven beter geïnformeerde keuzes maken over het gebruik van low-code ontwikkeling en wanneer traditionele codering beter aansluit.

\subsection{Low-code versus high-code}
Hoewel ze fundamenteel van elkaar verschillen, kunnen low-code en high-code ontwikkelmethoden elkaar aanvullen om aan diverse projecteisen te voldoen. High-code ontwikkeling levert ongeëvenaarde controle, flexibiliteit en de optie om complexe, aangepaste toepassingen vanaf nul te ontwikkelen, wat het perfect maakt voor grote of ingewikkelde projecten die afhankelijk zijn van geavanceerde functies en integraties. Low-code-platforms vergemakkelijken het ontwikkelproces met visuele hulpmiddelen en kant-en-klare componenten, wat ervoor zorgt dat prototypes en de ontwikkeling van middelmatig complexe toepassingen sneller verlopen.
\\
\\
Door deze methoden in evenwicht te brengen, kunnen bedrijven hun productiviteit maximaliseren, de time-to-market voor bepaalde projecten verkorten en ervaren ontwikkelaars inzetten waar ze het meest nodig zijn, zoals \textcite{Ballejos2024} benadrukt in uitgebreide overzicht van de verschillen en mogelijke wisselwerking tussen deze methodologieën.


% Voor literatuurverwijzingen zijn er twee belangrijke commando's:
% \autocite{KEY} => (Auteur, jaartal) Gebruik dit als de naam van de auteur
%   geen onderdeel is van de zin.
% \textcite{KEY} => Auteur (jaartal)  Gebruik dit als de auteursnaam wel een
%   functie heeft in de zin (bv. ``Uit onderzoek door Doll & Hill (1954) bleek
%   ...'')


%---------- Methodologie ------------------------------------------------------
\section{Methodologie}%
\label{sec:methodologie}
Dit onderzoek is opgedeeld in twee fasen: eerst het in kaart brengen van het probleemdomein (het ontbreken van een beslissingskader), gevolgd door onderzoek naar het oplossingsdomein (de ontwikkeling van het kader).
\subsection{Fase 1: Analyse van het probleemdomein}
Om een grondig inzicht te krijgen in het probleemdomein worden de volgende onderzoeksmethoden gebruikt:
\subsubsection{Analyse van historische projectdata}
Om deelvragen 1 en 2 te beantwoorden ("Wat zijn de gevolgen van het ontbreken van een \\ beslissingskader?" \; \hbox{en} \, "Welke problemen ontstaan er in projecten?"), wordt een gestructureerde analyse uitgevoerd van afgesloten projecten. Deze analyse omvat:
\begin{itemize}
    \item Vergelijking van initiële projectschattingen versus werkelijke uitkomsten om impact op kosten en doorlooptijd te kwantificeren
    \item Analyse van projectdocumentatie om momenten te identificeren waar keuzes tussen low-code en high-code tot problemen leidden
    \item Inventarisatie van situaties waarin late overschakeling naar alternatieve ontwikkelmethoden nodig was
    \item Evaluatie van de financiële en tijdsimpact van deze late aanpassingen
\end{itemize}
\subsubsection{Interviews met experts}
Voor het beantwoorden van deelvragen 3 en 4 ("Wat zijn de huidige criteria?" \; \hbox{en} \, "Welke knelpunten ervaren projectmanagers?") worden diepte-interviews gehouden met:
\begin{itemize}
    \item Projectmanagers over hun huidige besluitvormingsproces
    \item Architecten over hun ervaringen met technologie-keuzes
    \item Pre-sales consultants over hun aanpak bij het inschatten van projectgeschiktheid
    \item Ontwikkelaars over de uitdagingen die zij ervaren bij technologie-keuzes
\end{itemize}
\subsubsection{Documentatieonderzoek}
Om deelvraag 3 verder te onderbouwen wordt bestaande interne documentatie geanalyseerd:
\begin{itemize}
    \item Huidige richtlijnen en procedures voor projectaanpak
    \item Pre-sales documentatie en offertes
    \item Project kick-off documenten
    \item Architectuurbeslissingen en design documents
\end{itemize}
\subsection{Fase 2: Onderzoek naar het oplossingsdomein}
Na het volledig in kaart brengen van het probleem, richt het onderzoek zich op het ontwikkelen van een oplossing middels:
\subsubsection{Literatuuroverzicht}
Een diepgaande analyse van bestaande documentatie over Mendix en andere low-code platforms voor het beantwoorden van deelvraag 5 en 7:
\begin{itemize}
    \item Officiële Mendix productgidsen
    \item Casestudies van derden en rapporten uit de industrie
    \item Online forums en ontwikkelaarsgemeenschappen
    \item Academische publicaties over low-code ontwikkeling
\end{itemize}
\subsubsection{Praktisch onderzoek}
Uitvoering van praktische tests voor deelvraag 6 om de grenzen van het Mendix-platform te onderzoeken:
\begin{itemize}
    \item Schaalbaarheid en prestaties
    \item Integratiemogelijkheden
    \item Aanpasbaarheid en uitbreidbaarheid
    \item Ontwikkelingssnelheid
    \item Beveiliging en compliance
\end{itemize}
\subsubsection{Ontwikkeling beslissingskader}
Voor het beantwoorden van deelvraag 8 ("Welke best practices en richtlijnen voor de keuze tussen Mendix en traditionele ontwikkeling kunnen worden gedestilleerd uit de projectervaringen en literatuur?") wordt op basis van alle verzamelde inzichten een gestructureerd beslissingskader ontwikkeld dat projectmanagers en architecten ondersteunt bij het maken van weloverwogen keuzes over de inzet van Mendix versus traditionele ontwikkelmethoden.



%---------- Verwachte resultaten ----------------------------------------------
\section{Verwacht resultaat, conclusie}%
\label{sec:verwachte_resultaten}

Door het combineren van de kennis uit het literatuuronderzoek, de praktische bevindingen uit de experimenten en de inzichten uit de interviews met experts, heeft het onderzoek als doel een uitgebreid begrip te ontwikkelen van de mogelijkheden en beperkingen van Mendix in de context van de ontwikkeling van bedrijfsapplicaties. 
\\
\\
Als afsluiting van dit onderzoek wordt ook een beslissingskader opgesteld dat organisaties een gestructureerde aanpak biedt om een best passende keuze te maken tussen low-code en high-code ontwikkelmethoden. Dit kader stelt bedrijven zoals Apvine in staat om niet alleen effectiever te plannen, maar ook de risico's van late veranderingen in projecten te beperken. Hierdoor worden de ontwikkelingskosten beter beheerd en hoogwaardige oplossingen opgeleverd die aansluiten bij de unieke wensen van hun klanten.




    
    %%---------- Andere bijlagen --------------------------------------------------
    % TODO: Voeg hier eventuele andere bijlagen toe. Bv. als je deze BP voor de
    % tweede keer indient, een overzicht van de verbeteringen t.o.v. het origineel.
    %\input{...}
    
    %%---------- Backmatter, referentielijst ---------------------------------------
    
    \backmatter{}
    
    \setlength\bibitemsep{2pt} %% Add Some space between the bibliograpy entries
    \printbibliography[heading=bibintoc]

    
\end{document}
